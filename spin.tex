INTRO

The spin of an MBH is determined by several competing processes. The MBH accumulates most of its mass and angular momentum through accretion. Accretion from a gaseous disc will spin up the MBH, potentially leading to high spin values \citep{Volonteri2005}, while a series of randomly orientated accretion events will lead to a low spin value, and we expect an average value $|a| \sim 0.1$--$0.3 M_\bullet$ \citep{King2006, King2008}. The MBH will also grow through mergers. Minor mergers with smaller BHs can decrease the spin \citep{Hughes2003, Gammie2004}, while a series of major mergers, between similar mass MBHs, would lead to a likely spin of $|a| \sim 0.69 M_\bullet$ \citep{Berti2008, Berti2007, Gonzalez2007}. Measuring the spin of MBHs will give insight into the relative importance of these processes, and perhaps a glimpse into the histories of their host galaxies.

Elliptical and spiral galaxies are expected to host MBH's of differing spins because of their different evolutions: one expects MBHs in elliptical galaxies to have on average higher spin than black holes in spiral galaxies, where random, small accretion episodes might have played a more important role \citep{Volonteri2007, Sikora2007}.





PARAM

Naively this could be anywhere in the range $|a| < M_\bullet$; however it is possible to place an upper bound by considering spin-up mechanisms. Considering the torque from radiation emitted by an accretion disc, and swallowed by the BH, it can be shown that $|a| \lesssim 0.998 M_\bullet$ \citep{Thorne1974}. Magnetohydrodynamical simulations of accretion discs produce a smaller maximum value of $|a| \sim 0.95 M_\bullet$ \citep{Gammie2004}. The actual spin value could be much lower than this upper bound depending upon the MBH's evolution (as discussed in \secref{Intro}). We will use the convention that $a$ is positive, and will change the sense of rotation by flipping the $z$-axis.


  


