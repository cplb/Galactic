\documentclass[useAMS,usedcolumn,usegraphicx,usenatbib]{mn2e}
\usepackage{amssymb,amstext,amsfonts} %% ... with default font
\usepackage[fleqn]{amsmath}
\usepackage{subfigure}
\usepackage{slashed}
\usepackage[lowtilde]{url}
%\usepackage{enumitem}
\usepackage{mathrsfs}
\usepackage{dcolumn}
\usepackage{hyperref}

\renewcommand{\mathindent}{0cm}

%%%%% AUTHORS - PLACE YOUR OWN MACROS HERE %%%%%
\newcommand{\eqnref}[1]{(\ref{eq:#1})}
\newcommand{\figref}[1]{Fig.~\ref{fig:#1}}
\newcommand{\Figref}[1]{Figure~\ref{fig:#1}}
\newcommand{\tabref}[1]{Table~\ref{tab:#1}}
\newcommand{\secref}[1]{Sec.~\ref{sec:#1}}
\newcommand{\Secref}[1]{Section~\ref{sec:#1}}
\newcommand{\apref}[1]{Appendix~\ref{ap:#1}}

\newcommand{\units}[1]{\ensuremath{~\mathrm{#1}}}

\DeclareMathOperator{\sinc}{sinc}

\newcommand{\sub}[1]{\ensuremath{_\mathrm{#1}}}
\newcommand{\super}[1]{\ensuremath{^\mathrm{#1}}}
\newcommand{\dd}{\ensuremath{\mathrm{d}}}
\newcommand{\diff}[2]{\ensuremath{\frac{\dd {#1}}{\dd {#2}}}}
\newcommand{\partialdiff}[2]{\ensuremath{\frac{\partial {#1}}{\partial {#2}}}}
\newcommand{\intd}[4]{\ensuremath{\int_{#1}^{#2}{#3}\,\dd{#4}}}
\newcommand{\recip}[1]{\ensuremath{\frac{1}{#1}}}

\newcommand{\order}[1]{\ensuremath{\mathcal{O}({#1})}}

\newcommand{\innerprod}[2]{\ensuremath{\left({#1}\middle|{#2}\right)}}

\newcommand{\Ibar}{{\declareslashed{}{\text{-}}{0.04}{-0.2}{I}\slashed{I}}}
%%%%%%%%%%%%%%%%%%%%%%%%%%%%%%%%%%%%%%%%%%%%%%%%
\title[EMRBs from extragalactic sources]{Extreme-mass-ratio-bursts from extragalactic sources}
\author[C.\ P.\ L.\ Berry and J.\ R.\ Gair]{C.\ P.\ L.\ Berry$^{1}$\thanks{E-mail: cplb2@cam.ac.uk}  and J.\ R.\ Gair$^{1}$\\
$^{1}$Institute of Astronomy, University of Cambridge, Madingley Road, Cambridge, CB3 0HA}

\begin{document}

\date{\today}

\pagerange{\pageref{firstpage}--\pageref{lastpage}} \pubyear{2012}

\maketitle

\label{firstpage}

\begin{abstract}
An extreme-mass-ratio burst (EMRB) is a gravitational wave signal emitted when a compact object passes through periapsis on a highly eccentric orbit about a much more massive object, in our case a stellar mass object about a $10^6 M_\odot$ black hole. EMRBs are a relatively unexplored means of probing the spacetime of massive black holes (MBHs). We conduct an investigation of the properties of EMRBs and how they could allow us to constrain the parameters, such as its spin, of the Galaxy's MBH. We find that if an EMRB event occurs in the Galaxy, it should be detectable if the periapse distance is $r\sub{p} < 65 r\sub{g}$ for a $\mu = 10 M_\odot$ orbiting object, where $r\sub{g} = GM_\bullet/c^2$ is the gravitational radius. The signal-to-noise ratio scales approximately as $\log(\rho) \simeq -2.7\log(r\sub{p}/r\sub{g}) + \log(\mu/M_\odot) + 4.9$. For periapses smaller than $\sim 10 r\sub{g}$, EMRBs can be informative, and provide good constraints on both the MBH's mass and spin. Closer orbits provide better constraints, with the best giving accuracies of better than one part in $10^4$ for both the mass and spin parameter.
\end{abstract}

\begin{keywords}
black hole physics -- Galaxy: centre -- gravitational waves -- methods: data analysis.
\end{keywords}

\section{Introduction}\label{sec:Intro}

It is well established that space is big \citep{Adams}. The Milky Way, our own island universe, is but one of a multitude of galaxies. Each one of these may have a massive black hole (MBH) nestled at its core \citep{Lynden-Bell1971, Soltan1982}.

In previous work \citep{Berry2012}, we considered measuring the properties of the Galaxy's MBH using extreme-mass-ratio bursts (EMRBs). An EMRB is a short gravitational wave (GW) signal produced when a small object passes through periapsis on an orbit about a much more massive body; in our case this is a stellar mass compact object (CO) orbiting the MBH. If the periapse radius of the orbit is sufficiently small ($r\sub{p} \lesssim 10 r\sub{g}$ for a $10 M_\odot$ CO, where $r\sub{g} = GM_\bullet/c^2$ is a gravitational radius), a single burst can be highly informative about the MBH, improving our knowledge of its mass and spin.

EMRBs could be considered as the precursors to the better studied extreme-mass-ratio inspirals (EMRIs; \citealt{Amaro-Seoane2007}). Close collisions in the dense nuclear cluster surrounding the MBH scatter COs onto highly eccentric orbits. They proceed to emit an EMRB each orbit \citep*{Rubbo2006}. If they survive for long enough without being scattered again, the loss of energy-momentum carried away by gravitational radiation shall lead the orbit to circularise; eventually the GW signal changes, so there is continuous significant emission and we have an EMRI which continues until the inevitable plunge into the MBH. EMRBs are much shorter than EMRIs; they do not have as much time to accumulate high signal-to-noise ratios (SNRs), and consequently they are neither detectable to the same range, or as informative as EMRIs. However, a CO could emit many EMRBs before transitioning to the EMRI regime, making EMRBs an interesting signal for GW detection.

In this work, we consider if EMRBs are detectable from other nearby galaxies. If so, they may be useful for constraining the properties of these galaxies' MBHs. Observations have shown MBH masses are correlated with properties of the host galaxies, such as bulge luminosity, mass, velocity dispersion and light concentration \citep{Kormendy1995, Magorrian1998, Ferrarese2000, Gebhardt2000, Graham2001, Tremaine2002, Marconi2003, Haring2004, Graham2007, Graham2011}. The two are linked via their shared history, such that one can inform us about the other.

Astrophysical black holes (BHs) are described by two quantities: mass $M_\bullet$ and (dimsensionless) spin $a_\ast$ \citep{Chandrasekhar1998}. The spin is related to the angular momentum $J$ by
\begin{equation}
a_\ast = \frac{cJ}{GM_\bullet^2}.
\end{equation}
For many MBHs in the local neighbourhood, we have existing mass estimates. Measuring the spin would give us a complete picture, and would crucially give an insight into the formation history of the galaxy.

MBHs accumulate mass and angular momentum through accretion and mergers \citep{Volonteri2010, Yu2002}; the spin encodes information about the mechanism that has most recently dominated the evolution. A gaseous disc spins up the MBH, resulting in high spin values \citep{Volonteri2005}; randomly orientated accretion events lead to low spin values \citep*{King2006, King2008}; minor mergers with smaller BHs can decrease the spin \citep*{Hughes2003, Gammie2004}, and major mergers between MBHs gives a likely spin $|a_\ast| \sim 0.7$ \citep{Berti2008, Gonzalez2007}. Determining how the spin evolved shall tell us about how the galaxy evolved.

We have some MBH spin measurements from X-ray observations of active galactic nuclei \citep[e.g.]{Nardini2011, Patrick2011, Gallo2011, Lohfink2012}.  Estimates span the entire range of allowed values, but are typically in the intermediate range of $a_\ast \sim 0.7$, with uncertainties of $\sim 10\%$. This population would be an interesting comparison for 


An exciting means of inferring information about the MBH is through gravitational waves (GWs) emitted when compact objects (COs), such as stellar mass BHs, neutron stars (NSs), white dwarfs (WDs) or low mass main sequence (MS) stars, pass close by \citep{Sathyaprakash2009}. A space-borne detector, such as the \textit{Laser Interferometer Space Antenna} (\textit{LISA}) or the \textit{evolved Laser Interferometer Space Antenna} (\textit{eLISA}), is designed to be able to detect GWs in the frequency range of interest for these encounters \citep{Bender1998, Danzmann2003, Jennrich2011, Amaro-Seoane2012a}.\footnote{The revised \textit{eLISA} concept shares the same descoped design as the \textit{New Gravitational-wave Observatory} (\textit{NGO}) submitted to the European Space Agency for their L1 mission selection.} The identification of waves requires a set of accurate waveform templates covering parameter space. Much work has already been done on the waveforms generated when companion objects inspiral towards an MBH \citep{Glampedakis2005, Barack2009}; as they orbit, GWs carry away energy and angular momentum, causing the orbit to shrink until eventually the object plunges into the MBH. The initial orbits may be highly elliptical and a burst of radiation is emitted during each close encounter. These are extreme mass-ratio bursts (EMRBs; \citealt*{Rubbo2006}). Assuming that the companion is not scattered from its orbit, and does not plunge straight into the MBH, its orbit evolves, becoming more circular, and it shall begin to continuously emit significant gravitational radiation in the \textit{LISA}/\textit{eLISA} frequency range. The resulting signals are extreme mass-ratio inspirals (EMRIs; \citealt{Amaro-Seoane2007}).

Studies of these systems have usually focused upon the phase when the orbit is close to plunge and completes a large number of cycles in the detector's frequency band, allowing a high signal-to-noise ratio (SNR) to be accumulated. Here, we investigate high eccentricity orbits. These are the initial bursting orbits from which an EMRI may evolve, and are the consequence of scattering from two body encounters. The event rate for the detection of such EMRBs with \textit{LISA} has been estimated to be as high as $15\units{yr^{-1}}$ \citep{Rubbo2006}, although this has been subsequently revised downwards to the order of $1\units{yr^{-1}}$ \citep*{Hopman2007}. Even if only a single burst is detected during a mission, this is still an exciting possibility since the information carried by the GW should give an unparalleled probe of the structure of spacetime of the GC. Exactly what can be inferred depends upon the orbit, which is what we investigate here. 

We make the simplifying assumption that all these orbits are marginally bound, or parabolic, since highly eccentric orbits appear almost indistinguishable from an appropriate parabolic orbit. Here ``parabolic'' and ``eccentricity'' refer to the energy of the geodesic and not to the geometric shape of the orbit.\footnote{Marginally bound Keplerian orbits in flat spacetime are parabolic in both senses.} Following such a trajectory an object may make just one pass of the MBH or, if the periapsis distance is small enough, it may complete a number of rotations. Such an orbit is referred to as zoom-whirl \citep{Glampedakis2002a}.

In order to compute the gravitational waveform produced in such a case, we integrate the geodesic equations for a parabolic orbit in Kerr spacetime. We assume that the orbiting body is a test particle, such that it does not influence the underlying spacetime, and that the orbital parameters evolve negligibly during the orbit such that they may be held constant. We use this to construct an approximate numerical kludge (NK) waveform \citep{Babak2007}.

This paper is organised as follows. We begin in \secref{Geodesic} with the construction of the geodesic orbits; these trajectories are used in the construction of NK waveforms as explained in \secref{Kludge}. In \secref{Detector} we establish what the \textit{LISA} detectors would measure, and in \secref{Signal} how the signal would be analysed. This includes a brief mention of window functions which is expanded in \apref{window}. Here we also present a novel window function, the Planck-Bessel window, which may be of use for signals with a large dynamic range. In \secref{Waveforms} we look at our NK waveforms. We give fiducial power-law fits for SNR as a function of periapse radius, which may be of use for back-of-the-envelope estimates. We confirm the accuracy of the kludge waveforms in \secref{Energy} by comparing the energy flux to fluxes calculated using other approaches. The typical error introduced by the NK approximation may be a few percent, but this worsens as the periapsis approaches the last non-plunging orbit. We explain how to extract the information from the bursts in \secref{Estimation}. Results estimating the precision to which parameters could be measured are presented in \secref{Results}. We briefly mention the possibility of detecting bursts from extra-galactic sources in \secref{Extragal}, before concluding in \secref{End} with a summary of our results. EMRBs may be informative if the event rate is high enough for them to be a viable source.

There are currently no funded space-borne detector missions. The \textit{eLISA} mission concept remains an active field of study. It is hoped to submit this to the European Space Agency as a potential cornerstone mission. We use the classic \textit{LISA} design for this work. This is done from historical affection in lieu of a definite alternative. Should funding for a space-borne detector be secured in the future it is hoped that it shall have comparable sensitivity to \textit{LISA}, and that studies using the \textit{LISA} design shall be a sensible benchmark for comparison. We find that to obtain good results the periapse radius must be $r\sub{p} \lesssim 10 r\sub{g}$, where $r\sub{g} = GM_\bullet / c^2$ is a gravitational radius; at this point the SNR is already high: for parameter estimation the orbit is more important that the signal strength, and so the exact detector performance should be of secondary importance.

Throughout this work we adopt a metric with signature $(+,-,-,-)$. Greek indices are used to represent spacetime indices $\mu = \{0,1,2,3\}$ and lowercase Latin indices from the middle of the alphabet are used for spatial indices $i = \{1,2,3\}$. Uppercase Latin indices from the beginning of the alphabet are used for the output of the two \textit{LISA} detector-arms $A = \{\mathrm{I}, \mathrm{II}\}$, and lowercase Latin indices from the beginning of the alphabet are used for parameter space. Summation over repeated indices is assumed unless explicitly noted otherwise. Geometric units with $G = c = 1$ are used where noted, but in general factors of $G$ and $c$ are retained.

\section*{Acknowledgments}

The authors would like to thank Donald Lynden-Bell for useful converstaions. CPLB is supported by STFC. JRG is supported by the Royal Society. The MCMC simulations were performed using the Darwin Supercomputer of the University of Cambridge High Performance Computing Service (\url{http://www.hpc.cam.ac.uk/}), provided by Dell Inc.\ using Strategic Research Infrastructure Funding from the Higher Education Funding Council for England.

\bibliographystyle{mn3e}
\bibliography{Extragalactic}

\bsp

\label{lastpage}

\end{document}
