\documentclass[useAMS,usedcolumn,usegraphicx,usenatbib]{mn2e}
\usepackage{amssymb,amstext,amsfonts} %% ... with default font
\usepackage[fleqn]{amsmath}
\usepackage{subfigure}
\usepackage{slashed}
\usepackage[lowtilde]{url}
%\usepackage{enumitem}
\usepackage{mathrsfs}
\usepackage{dcolumn}
\usepackage{hyperref}

\renewcommand{\mathindent}{0cm}

%%%%% AUTHORS - PLACE YOUR OWN MACROS HERE %%%%%
\newcommand{\eqnref}[1]{(\ref{eq:#1})}
\newcommand{\figref}[1]{Fig.~\ref{fig:#1}}
\newcommand{\Figref}[1]{Figure~\ref{fig:#1}}
\newcommand{\tabref}[1]{Table~\ref{tab:#1}}
\newcommand{\secref}[1]{Sec.~\ref{sec:#1}}
\newcommand{\Secref}[1]{Section~\ref{sec:#1}}
\newcommand{\apref}[1]{Appendix~\ref{ap:#1}}

\newcommand{\units}[1]{\ensuremath{~\mathrm{#1}}}

\DeclareMathOperator{\sinc}{sinc}

\newcommand{\sub}[1]{\ensuremath{_\mathrm{#1}}}
\newcommand{\super}[1]{\ensuremath{^\mathrm{#1}}}
\newcommand{\dd}{\ensuremath{\mathrm{d}}}
\newcommand{\diff}[2]{\ensuremath{\frac{\dd {#1}}{\dd {#2}}}}
\newcommand{\partialdiff}[2]{\ensuremath{\frac{\partial {#1}}{\partial {#2}}}}
\newcommand{\intd}[4]{\ensuremath{\int_{#1}^{#2}{#3}\,\dd{#4}}}
\newcommand{\recip}[1]{\ensuremath{\frac{1}{#1}}}

\newcommand{\order}[1]{\ensuremath{\mathcal{O}({#1})}}

\newcommand{\innerprod}[2]{\ensuremath{\left({#1}\middle|{#2}\right)}}

\newcommand{\Ibar}{{\declareslashed{}{\text{-}}{0.04}{-0.2}{I}\slashed{I}}}
%%%%%%%%%%%%%%%%%%%%%%%%%%%%%%%%%%%%%%%%%%%%%%%%
\title[EMRBs from extragalctic sourses]{Extreme-mass-ratio-bursts from extragalactic sources}
\author[C.\ P.\ L.\ Berry and J.\ R.\ Gair]{C.\ P.\ L.\ Berry$^{1}$\thanks{E-mail: cplb2@cam.ac.uk}  and J.\ R.\ Gair$^{1}$\\
$^{1}$Institute of Astronomy, University of Cambridge, Madingley Road, Cambridge, CB3 0HA}

\begin{document}

\date{\today}

\pagerange{\pageref{firstpage}--\pageref{lastpage}} \pubyear{2012}

\maketitle

\label{firstpage}

\begin{abstract}
An extreme-mass-ratio burst (EMRB) is a gravitational wave signal emitted when a compact object passes through periapsis on a highly eccentric orbit about a much more massive object, in our case a stellar mass object about a $10^6 M_\odot$ black hole. EMRBs are a relatively unexplored means of probing the spacetime of massive black holes (MBHs). We conduct an investigation of the properties of EMRBs and how they could allow us to constrain the parameters, such as its spin, of the Galaxy's MBH. We find that if an EMRB event occurs in the Galaxy, it should be detectable if the periapse distance is $r\sub{p} < 65 r\sub{g}$ for a $\mu = 10 M_\odot$ orbiting object, where $r\sub{g} = GM_\bullet/c^2$ is the gravitational radius. The signal-to-noise ratio scales approximately as $\log(\rho) \simeq -2.7\log(r\sub{p}/r\sub{g}) + \log(\mu/M_\odot) + 4.9$. For periapses smaller than $\sim 10 r\sub{g}$, EMRBs can be informative, and provide good constraints on both the MBH's mass and spin. Closer orbits provide better constraints, with the best giving accuracies of better than one part in $10^4$ for both the mass and spin parameter.
\end{abstract}

\begin{keywords}
black hole physics -- Galaxy: centre -- gravitational waves -- methods: data analysis.
\end{keywords}

\section{Introduction}\label{sec:Intro}





Many, if not all, galactic nuclei have harboured a massive black hole (MBH) during their evolution \citep{Lynden-Bell1971, Soltan1982, Rees1984}. Observations have shown that there exist well-defined correlations between the MBHs' masses and the properties of their host galaxies, such as bulge luminosity, mass, velocity dispersion and light concentration \citep{Kormendy1995, Magorrian1998, Ferrarese2000, Gebhardt2000, Graham2001, Tremaine2002, Marconi2003, Haring2004, Graham2007, Graham2011}. These suggest coeval evolution of the MBH and galaxy \citep{Peng2007, Jahnke2011}, possibly with feedback mechanisms coupling the two \citep{Haiman2004, Volonteri2009}. The MBH and the surrounding spheroidal component share a common history, such that the growth of one can inform us about the growth of the other.

The best opportunity to study MBHs comes from the compact object in our own galactic centre (GC), which is coincident with Sagittarius A* (Sgr A*). Through careful monitoring of stars orbiting the GC, this has been identified as an MBH of mass $M_\bullet = 4.31 \times 10^6 M_\odot$ at a distance of only $R_0 = 8.33\units{kpc}$ \citep{Gillessen2009}.

According to the no-hair theorem, the MBH should be described completely by just its mass $M_\bullet$ and spin $a$, since we expect the charge of an astrophysical black hole to be negligible \citep{Israel1967, Israel1968, Carter1971, Hawking1972, Robinson1975, Chandrasekhar1998}. The spin parameter $a$ is related to the BH's angular momentum $J$ by
\begin{equation}
J = M_\bullet ac;
\end{equation}
it is often convenient to use the dimensionless spin
\begin{equation}
a_\ast = \frac{cJ}{GM_\bullet^2}.
\end{equation}
As we have a good estimate of the mass, to gain a complete description of the MBH we have only to measure its spin; this shall give us insight into its history and role in the evolution of the Galaxy.

The spin of an MBH is determined by several competing processes. An MBH accumulates mass and angular momentum through accretion \citep{Volonteri2010}. Accretion from a gaseous disc shall spin up the MBH, potentially leading to high spin values \citep{Volonteri2005}, while a series of randomly orientated accretion events leads to a low spin value: we expect an average value $|a_\ast| \sim 0.1$--$0.3$ \citep*{King2006, King2008}. The MBH also grows through mergers \citep{Yu2002, Malbon2007}. Minor mergers with smaller black holes (BHs) can decrease the spin \citep*{Hughes2003, Gammie2004}, while a series of major mergers, between similar mass MBHs, would lead to a likely spin of $|a_\ast| \sim 0.69$ \citep{Berti2008, Berti2007, Gonzalez2007}. Measuring the spin of MBHs shall help us understand the relative importance of these processes, and perhaps gain a glimpse into their host galaxies' pasts.

Elliptical and spiral galaxies are believed to host MBHs of differing spins because of their different evolutions: we expect MBHs in elliptical galaxies to have on average higher spins than black holes in spiral galaxies, where random, small accretion episodes have played a more important role \citep*{Volonteri2007, Sikora2007}.

It has been suggested that the spin of the Galaxy's MBH could be inferred from careful observation of the orbits of stars within a few milliparsecs of the GC \citep{Merritt2010}, although this is complicated because of perturbations due to other stars, or from observations of quasi-periodic oscillations in the luminosity of flares believed to originate from material orbiting close to the innermost stable orbits \citep{Genzel2003a, Belanger2006, Trippe2007, Hamaus2009, Kato2010}, though there are difficulties in interpreting these results \citep{Psaltis2008a}.

This latter method, combined with a disc-seismology model, has produced a value of the dimensionless spin of $a_\ast = 0.44 \pm 0.08$. To obtain this result \citet{Kato2010} have combined their observations of Sgr A* with observations of galactic X-ray sources containing solar mass BHs, to find a best-fit unique spin parameter for all BHs. However, it is not clear that all BHs should share the same value of the spin parameter; especially considering that the BHs considered here differ in mass by six orders of magnitude, with none in the intermediate range. Even if BH spin is determined by a universal process, we still expect some distribution of spin parameters \citep{King2008, Berti2008}. Thus we cannot precisely determine the spin of the galactic centre's MBH from an average including other BHs.

The spins of MBHs in active galactic nuclei have been inferred using X-ray observations of $\mathrm{Fe}$ $\mathrm{K}$ emission lines \citep{Miller2007, McClintock2011}. So far this has been done for a handful of other galaxies' central MBHs \citep{Brenneman2006, Miniutti2009, Schmoll2009, delaCallePerez2010, Zoghbi2010, Nardini2011,  Patrick2011}. Estimates for the spin cover a range of values up to the maximal value for an extremal Kerr black hole. Typical values are in the intermediate range of $a_\ast \sim 0.7$ with an uncertainty of about $10\%$ on each measurement.

While we can use the spin of other BHs as a prior, to inform us of what we should expect to measure for the spin of the Galaxy's MBH, it is desirable to have an independent observation, a direct measurement.

An exciting means of inferring information about the MBH is through gravitational waves (GWs) emitted when compact objects (COs), such as stellar mass BHs, neutron stars (NSs), white dwarfs (WDs) or low mass main sequence (MS) stars, pass close by \citep{Sathyaprakash2009}. A space-borne detector, such as the \textit{Laser Interferometer Space Antenna} (\textit{LISA}) or the \textit{evolved Laser Interferometer Space Antenna} (\textit{eLISA}), is designed to be able to detect GWs in the frequency range of interest for these encounters \citep{Bender1998, Danzmann2003, Jennrich2011, Amaro-Seoane2012a}.\footnote{The revised \textit{eLISA} concept shares the same descoped design as the \textit{New Gravitational-wave Observatory} (\textit{NGO}) submitted to the European Space Agency for their L1 mission selection.} The identification of waves requires a set of accurate waveform templates covering parameter space. Much work has already been done on the waveforms generated when companion objects inspiral towards an MBH \citep{Glampedakis2005, Barack2009}; as they orbit, GWs carry away energy and angular momentum, causing the orbit to shrink until eventually the object plunges into the MBH. The initial orbits may be highly elliptical and a burst of radiation is emitted during each close encounter. These are extreme mass-ratio bursts (EMRBs; \citealt*{Rubbo2006}). Assuming that the companion is not scattered from its orbit, and does not plunge straight into the MBH, its orbit evolves, becoming more circular, and it shall begin to continuously emit significant gravitational radiation in the \textit{LISA}/\textit{eLISA} frequency range. The resulting signals are extreme mass-ratio inspirals (EMRIs; \citealt{Amaro-Seoane2007}).

Studies of these systems have usually focused upon the phase when the orbit is close to plunge and completes a large number of cycles in the detector's frequency band, allowing a high signal-to-noise ratio (SNR) to be accumulated. Here, we investigate high eccentricity orbits. These are the initial bursting orbits from which an EMRI may evolve, and are the consequence of scattering from two body encounters. The event rate for the detection of such EMRBs with \textit{LISA} has been estimated to be as high as $15\units{yr^{-1}}$ \citep{Rubbo2006}, although this has been subsequently revised downwards to the order of $1\units{yr^{-1}}$ \citep*{Hopman2007}. Even if only a single burst is detected during a mission, this is still an exciting possibility since the information carried by the GW should give an unparalleled probe of the structure of spacetime of the GC. Exactly what can be inferred depends upon the orbit, which is what we investigate here. 

We make the simplifying assumption that all these orbits are marginally bound, or parabolic, since highly eccentric orbits appear almost indistinguishable from an appropriate parabolic orbit. Here ``parabolic'' and ``eccentricity'' refer to the energy of the geodesic and not to the geometric shape of the orbit.\footnote{Marginally bound Keplerian orbits in flat spacetime are parabolic in both senses.} Following such a trajectory an object may make just one pass of the MBH or, if the periapsis distance is small enough, it may complete a number of rotations. Such an orbit is referred to as zoom-whirl \citep{Glampedakis2002a}.

In order to compute the gravitational waveform produced in such a case, we integrate the geodesic equations for a parabolic orbit in Kerr spacetime. We assume that the orbiting body is a test particle, such that it does not influence the underlying spacetime, and that the orbital parameters evolve negligibly during the orbit such that they may be held constant. We use this to construct an approximate numerical kludge (NK) waveform \citep{Babak2007}.

This paper is organised as follows. We begin in \secref{Geodesic} with the construction of the geodesic orbits; these trajectories are used in the construction of NK waveforms as explained in \secref{Kludge}. In \secref{Detector} we establish what the \textit{LISA} detectors would measure, and in \secref{Signal} how the signal would be analysed. This includes a brief mention of window functions which is expanded in \apref{window}. Here we also present a novel window function, the Planck-Bessel window, which may be of use for signals with a large dynamic range. In \secref{Waveforms} we look at our NK waveforms. We give fiducial power-law fits for SNR as a function of periapse radius, which may be of use for back-of-the-envelope estimates. We confirm the accuracy of the kludge waveforms in \secref{Energy} by comparing the energy flux to fluxes calculated using other approaches. The typical error introduced by the NK approximation may be a few percent, but this worsens as the periapsis approaches the last non-plunging orbit. We explain how to extract the information from the bursts in \secref{Estimation}. Results estimating the precision to which parameters could be measured are presented in \secref{Results}. We briefly mention the possibility of detecting bursts from extra-galactic sources in \secref{Extragal}, before concluding in \secref{End} with a summary of our results. EMRBs may be informative if the event rate is high enough for them to be a viable source.

There are currently no funded space-borne detector missions. The \textit{eLISA} mission concept remains an active field of study. It is hoped to submit this to the European Space Agency as a potential cornerstone mission. We use the classic \textit{LISA} design for this work. This is done from historical affection in lieu of a definite alternative. Should funding for a space-borne detector be secured in the future it is hoped that it shall have comparable sensitivity to \textit{LISA}, and that studies using the \textit{LISA} design shall be a sensible benchmark for comparison. We find that to obtain good results the periapse radius must be $r\sub{p} \lesssim 10 r\sub{g}$, where $r\sub{g} = GM_\bullet / c^2$ is a gravitational radius; at this point the SNR is already high: for parameter estimation the orbit is more important that the signal strength, and so the exact detector performance should be of secondary importance.

Throughout this work we adopt a metric with signature $(+,-,-,-)$. Greek indices are used to represent spacetime indices $\mu = \{0,1,2,3\}$ and lowercase Latin indices from the middle of the alphabet are used for spatial indices $i = \{1,2,3\}$. Uppercase Latin indices from the beginning of the alphabet are used for the output of the two \textit{LISA} detector-arms $A = \{\mathrm{I}, \mathrm{II}\}$, and lowercase Latin indices from the beginning of the alphabet are used for parameter space. Summation over repeated indices is assumed unless explicitly noted otherwise. Geometric units with $G = c = 1$ are used where noted, but in general factors of $G$ and $c$ are retained.

\section*{Acknowledgments}

The authors would like to thank Donald Lynden-Bell for useful converstaions. CPLB is supported by STFC. JRG is supported by the Royal Society. The MCMC simulations were performed using the Darwin Supercomputer of the University of Cambridge High Performance Computing Service (\url{http://www.hpc.cam.ac.uk/}), provided by Dell Inc.\ using Strategic Research Infrastructure Funding from the Higher Education Funding Council for England.

\bibliographystyle{mn3e}
\bibliography{Extragalactic}

\bsp

\label{lastpage}

\end{document}
