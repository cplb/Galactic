\documentclass[useAMS,usedcolumn,usegraphicx,usenatbib]{mn2e}
\usepackage{amssymb,amstext,amsfonts} %% ... with default font
\usepackage[fleqn]{amsmath}
\usepackage{subfigure}
\usepackage{slashed}
\usepackage[lowtilde]{url}
%\usepackage{enumitem}
\usepackage{mathrsfs}
\usepackage{dcolumn}
\usepackage{hyperref}

\renewcommand{\mathindent}{0cm}

%%%%% AUTHORS - PLACE YOUR OWN MACROS HERE %%%%%
\newcommand{\eqnref}[1]{(\ref{eq:#1})}
\newcommand{\figref}[1]{Fig.~\ref{fig:#1}}
\newcommand{\Figref}[1]{Figure~\ref{fig:#1}}
\newcommand{\tabref}[1]{Table~\ref{tab:#1}}
\newcommand{\secref}[1]{Sec.~\ref{sec:#1}}
\newcommand{\Secref}[1]{Section~\ref{sec:#1}}
\newcommand{\apref}[1]{Appendix~\ref{ap:#1}}

\newcommand{\units}[1]{\ensuremath{~\mathrm{#1}}}

\DeclareMathOperator{\sinc}{sinc}

\newcommand{\sub}[1]{\ensuremath{_\mathrm{#1}}}
\newcommand{\super}[1]{\ensuremath{^\mathrm{#1}}}
\newcommand{\dd}{\ensuremath{\mathrm{d}}}
\newcommand{\diff}[2]{\ensuremath{\frac{\dd {#1}}{\dd {#2}}}}
\newcommand{\partialdiff}[2]{\ensuremath{\frac{\partial {#1}}{\partial {#2}}}}
\newcommand{\intd}[4]{\ensuremath{\int_{#1}^{#2}{#3}\,\dd{#4}}}
\newcommand{\recip}[1]{\ensuremath{\frac{1}{#1}}}

\newcommand{\order}[1]{\ensuremath{\mathcal{O}({#1})}}

\newcommand{\innerprod}[2]{\ensuremath{\left({#1}\middle|{#2}\right)}}

\newcommand{\Ibar}{{\declareslashed{}{\text{-}}{0.04}{-0.2}{I}\slashed{I}}}
%%%%%%%%%%%%%%%%%%%%%%%%%%%%%%%%%%%%%%%%%%%%%%%%
\title[EMRBs from extragalactic sources]{Extreme-mass-ratio-bursts from extragalactic sources}
\author[C.\ P.\ L.\ Berry and J.\ R.\ Gair]{C.\ P.\ L.\ Berry$^{1}$\thanks{E-mail: cplb2@cam.ac.uk}  and J.\ R.\ Gair$^{1}$\\
$^{1}$Institute of Astronomy, University of Cambridge, Madingley Road, Cambridge, CB3 0HA}

\begin{document}

\date{\today}

\pagerange{\pageref{firstpage}--\pageref{lastpage}} \pubyear{2012}

\maketitle

\label{firstpage}

\begin{abstract}
Extreme-mass-ratios bursts (EMRBs) are a potentially interesting gravitational wave signal. They are produced when a compact object passes through periapsis on a highly eccentric orbit about a much more massive object; we consider stellar mass objects orbiting the massive black holes (MBHs) found in the centre of galaxies. Such an object may emit many EMRBs before eventually inspiralling into the MBH. EMRBs from the Galaxy's MBH would be detectable with a space-borne gravitational wave detector. We investigate the possibility of detecting EMRBs from extragalactic sources. 

\end{abstract}

\begin{keywords}
black hole physics -- Galaxy: centre -- gravitational waves -- methods: data analysis.
\end{keywords}

\section{Introduction}\label{sec:Intro}

It is well established that space is big \citep[chapter 8]{Adams1979}. The Milky Way, our own island universe, is but one of a multitude of galaxies. Each one of these may have a massive black hole (MBH) nestled at its core \citep{Lynden-Bell1971, Soltan1982}.

In previous work \citep{Berry2012}, we considered measuring the properties of the Galaxy's MBH using extreme-mass-ratio bursts (EMRBs). An EMRB is a short gravitational wave (GW) signal produced when a small object passes through periapsis on an orbit about a much more massive body; in our case this is a stellar mass compact object (CO) orbiting the MBH. If the periapse radius of the orbit is sufficiently small ($r\sub{p} \lesssim 10 r\sub{g}$ for a $10 M_\odot$ CO, where $r\sub{g} = GM_\bullet/c^2$ is a gravitational radius), a single burst can be highly informative about the MBH, improving our knowledge of its mass and spin.

EMRBs could be considered as the precursors to the better studied extreme-mass-ratio inspirals (EMRIs; \citealt{Amaro-Seoane2007}). Close collisions in the dense nuclear cluster surrounding the MBH scatter COs onto highly eccentric orbits. They proceed to emit an EMRB each orbit \citep*{Rubbo2006}. If they survive for long enough without being scattered again, the loss of energy-momentum carried away by gravitational radiation shall lead the orbit to circularise; eventually the GW signal changes, so there is continuous significant emission and we have an EMRI which continues until the inevitable plunge into the MBH. EMRBs are much shorter than EMRIs; they do not have as much time to accumulate high signal-to-noise ratios (SNRs), and consequently they are neither detectable to the same range, or as informative as EMRIs. However, a CO could emit many EMRBs before transitioning to the EMRI regime, making EMRBs an interesting signal for GW detection.

In this work, we consider if EMRBs are detectable from other nearby galaxies. If so, they may be useful for constraining the properties of these galaxies' MBHs. Observations have shown MBH masses are correlated with properties of the host galaxies, such as bulge luminosity, mass, velocity dispersion and light concentration \citep{Kormendy1995, Magorrian1998, Ferrarese2000, Gebhardt2000, Graham2001, Tremaine2002, Marconi2003, Haring2004, Graham2007, Graham2011}. The two are linked via their shared history, such that one can inform us about the other.

Astrophysical black holes (BHs) are described by two quantities: mass $M$ and (dimsensionless) spin $a_\ast$ \citep{Chandrasekhar1998}. The spin is related to the angular momentum $J$ by
\begin{equation}
a_\ast = \frac{cJ}{GM^2},
\end{equation}
For many MBHs in the local neighbourhood, we have existing mass estimates. Measuring the spin would give us a complete picture, and would crucially give an insight into the formation history of the galaxy\citep{Dotti2012,Volonteri2012a}.

MBHs accumulate mass and angular momentum through accretion and mergers \citep{Volonteri2010, Yu2002}; the spin encodes information about the mechanism that has most recently dominated the evolution. A gaseous disc spins up the MBH, resulting in high spin values \citep{Volonteri2005}; randomly orientated accretion events lead to low spin values \citep*{King2006, King2008}; minor mergers with smaller BHs can decrease the spin \citep*{Hughes2003, Gammie2004}, and major mergers between MBHs gives a likely spin $|a_\ast| \sim 0.7$ \citep{Berti2008, Gonzalez2007}. Determining how the spin evolved shall tell us about how the galaxy evolved\citep{Barausse2012}.

We have some MBH spin measurements from X-ray observations of active galactic nuclei \citep[e.g.]{Nardini2011, Patrick2011, Gallo2011, Lohfink2012}.  Estimates span the entire range of allowed values, but are typically in the intermediate range of $a_\ast \sim 0.7$, with uncertainties of $\sim 10\%$. This population would be an interesting comparison for potential measurements from nearby galaxies.

EMRBs could be an interesting signal for a space-borne gravitational wave detector, such as the \textit{Laser Interferometer Space Antenna} (\textit{LISA}; \citealt{Bender1998, Danzmann2003}) or the \textit{evolved Laser Interferometer Space Antenna} (\textit{eLISA}; \citealt{Jennrich2011, Amaro-Seoane2012a}).\footnote{The revised \textit{eLISA} concept is the same revised design as the \textit{New Gravitational-wave Observatory} (\textit{NGO}) submitted to the European Space Agency for their L1 mission selection.} At the time of writing, there is no currently funded mission. However, \textit{LISA Pathfinder}, a technology demonstration mission, is due for launch at the end of 2014 \citep{Anza2005, Antonucci2012}. Hopefully, a full mission shall follow in the subsequent decade. Since there does not exist a definite mission design, we stick to the classic \textit{LISA} design for the majority of this work.

EMRB waveforms are calculated and analysed as in \citet{Berry2012}, and we give only an outline of the techniques used. Waveform construction and the numerical kludge approximation are explained in \secref{Wave}. The basics of signal analysis are introduced in \secref{Sig}. In \secref{SNR}, the detectability of EMRBs from extragalactic MBHs is discussed. We show that bursts from other galaxies could be detected with \textit{LISA} or \textit{eLISA}. Following this, in \secref{MCMC} we discuss the information that could be extracted from these signals, and the constraints these could place.

Throughout this work we adopt a metric with signature $(+,-,-,-)$. Greek indices are used to represent spacetime indices $\mu = \{0,1,2,3\}$ and lowercase Latin indices from the middle of the alphabet are used for spatial indices $i = \{1,2,3\}$. Lowercase Latin indices from the beginning of the alphabet are used for parameter space. Summation over repeated indices is assumed unless explicitly noted otherwise. Geometric units with $G = c = 1$ are used where noted, but in general factors of $G$ and $c$ are retained.

\section{Waveform generation}\label{sec:Wave}

We employ a semirelativistic approximation \citep{Ruffini1981}: the CO travels along a geodesic in Kerr spacetime, but radiates as if it were in flat spacetime. This approach is known as a numerical kludge (NK). Comparison with more accurate, and computationally intensive, methods has shown that NK waveforms are reasonably accurate for extreme-mass-ratio systems \citep{Gair2005, Babak2007}: typical errors can few percent \citep{Tanaka1993,Gair2005,Berry2012}. Binding the motion to a true geodesic ensures the signal has the correct frequency components, although neglecting the effects of background curvature ensures that these do not have the correct amplitudes. The geodesic parameters are kept fixed throughout the orbit, as there should be negligible evolution due to the emission of gravitational radiation.

All bursts are assumed to come from marginally bound, or parabolic, orbits. In this case, the CO starts at rest at infinity and has a single passage through periapsis. If the periapse radius is small enough, the orbit may still complete a number of rotations about the MBH; these are zoom-whirl orbits \citep{Glampedakis2002a}.

When integrating the Kerr geodesic equations, we use angular variables instead of the radial and polar Boyer-Lindquist coordinates \citep{Drasco2004}
\begin{align}
r = {} & \frac{2 r\sub{p}}{1 + \cos\psi};\\
\cos^2\theta = {} & \frac{Q}{Q+L_z^2}\cos^2\chi = \sin^2 \iota \cos^2\chi,
\end{align}
where $Q$ is the Carter constant, $L_z$ is the angular momentum about the $z$-axis and $\iota$ is the orbital inclination \citep*{Glampedakis2002}. This parametrization avoids complications associated with turning points of the motion.

Once the geodesic is constructed, we identify the Boyer-Lindquist co-ordinates with flat-space spherical polars \citep{Gair2005, Babak2007}. This choice is not unique, as a consequence of the arbitrary nature of the NK approximation. Using flat-space oblate spheroidal coordinates gives quantitatively similar results \citep{Berry2012}. The quadrupole-octupole formula is used to derive the gravitational strain \citep{Bekenstein1973, Press1977, Yunes2008}. The inclusion of higher order terms modify the amplitudes of some frequency components for the more relativistic orbits by a few tens of percent.

The waveform is specified by a set of 14 parameters:
\begin{enumerate}
\item[(1)] The MBH's mass $M$.
\item[(2)] The spin parameter $a_\ast$.
\item[(3, 4)] The orientation angles for the MBH $\Theta\sub{K}$ and $\Phi\sub{K}$.
\item[(5)] The ratio of the SS-GC distance $R$ and the CO mass $\mu$, which we denote as $\zeta = R/\mu$. This scales the amplitude of the waveform.
\item[(6, 7)] The angular momentum of the CO parametrised in terms of total angular momentum $L_\infty = sqrt{Q + L_Z^2}$ and inclination $\iota$. We employ the latter, as the total angular momentum and inclination are less tightly correlated. Assuming spherical symmetry, we expect $\cos \iota$ to be uniformly distributed.
\item[(8--10)] A set of coordinates to specify the trajectory. We use the angular phases at periapse, $\phi\sub{p}$ and $\chi\sub{p}$ (which determines $\theta\sub{p}$), as well as the time of periapse $t\sub{p}$.
\item[(11, 12)] The coordinates of the MBH from the SS barycentre $\overline{\Theta}$ and $\overline{\Phi}$. These may be taken as the coordinates of Sgr A*, as the radio source is expected to be within $20 r\sub{g}$ of the MBH \citep{Reid2003,Doeleman2008}. We use the J2000.0 coordinates \citep{Reid1999, Yusef-Zadeh1999}. These change with time due to the rotation of the SS about the GC; the proper motion is about $6\units{mas\,yr^{-1}}$, mostly in the plane of the Galaxy \citep{Backer1999, Reid2003}. The position is already determined to high accuracy: an EMRB can only give weak constraints on source position.\footnote{For comparison, an EMRI, which should be more informative, can only give sky localisation to $\sim10^{-3}~\mathrm{steradians}$ \citep{Barack2004, Huerta2009}.} We take it as known and do not try to infer it.
\item[(13, 14)] The orbital position of the detector given by $\overline{\phi}$ and $\varphi$. We assume the initial positions are chosen such that $\overline{\phi} = 0$ when $\varphi = 0$ \citep{Cutler1998}; this choice does not qualitatively influence our results. The orbital position should be known, so need not be inferred.
\end{enumerate}

\section{Signal analysis}\label{sec:Sig}

In this section we briefly cover the basics of GW analysis. A more complete discussion can be found in \citet{Finn1992} and \citet{Cutler1994}. Those familiar with the subject may skip this section with impunity. In the following, uppercase Latin indices from the beginning of the alphabet are used for labelling detectors: $A = \{\mathrm{I}, \mathrm{II}\}$ for \textit{LISA}, but $A = \{\mathrm{I}\}$ for \textit{eLISA} which only has two arms, and so acts as a single detector.

The measured strain $\boldsymbol{s}(t)$ is the combination of the signal and the detector noise
\begin{equation}
\boldsymbol{s}(t) = \boldsymbol{h}(t) + \boldsymbol{n}(t);
\end{equation}
we assume the noise is stationary and Gaussian, and noise in multiple data channels uncorrelated, but shares the same characterisation \citep{Cutler1998}. We can then define a signal inner product\citep{Cutler1994}
\begin{equation}
\innerprod{\boldsymbol{g}}{\boldsymbol{k}} = 2\intd{0}{\infty}{\frac{\tilde{g}_A^\ast(f)\tilde{k}_A(f) + \tilde{g}_A(f)\tilde{k}_A^\ast(f)}{S\sub{n}(f)}}{f},
\label{eq:inner}
\end{equation}
introducing Fourier transforms
\begin{equation}
\tilde{g}(f) = \mathscr{F}\{g(t)\} = \intd{-\infty}{\infty}{g(t)\exp(2\pi i ft)}{t},
\end{equation}
and $S\sub{n}(f)$ is the noise spectral density. We use the noise model of \citet{Barack2004} for \textit{LISA}, and the simplified sensitivity model from \citet{Jennrich2011} for \textit{eLISA}.

The signal-to-noise ratio (SNR) is
\begin{equation}
\rho[\boldsymbol{h}] = \innerprod{\boldsymbol{h}}{\boldsymbol{h}}^{1/2}.
\label{eq:SNR}
\end{equation}
The probability of a realization of noise $\boldsymbol{n}(t) = \boldsymbol{n}_0(t)$ is
\begin{equation}
p(\boldsymbol{n}(t) = \boldsymbol{n}_0(t)) \propto \exp\left[-\recip{2}\innerprod{\boldsymbol{n}_0}{\boldsymbol{n}_0}\right].
\end{equation}
Therefore, if the incident waveform is $\boldsymbol{h}(t)$, the probability of measuring signal $\boldsymbol{s}(t)$ is
\begin{equation}
p(\boldsymbol{s}(t)|\boldsymbol{h}(t)) \propto \exp\left[-\recip{2}\innerprod{\boldsymbol{s}-\boldsymbol{h}}{\boldsymbol{s}-\boldsymbol{h}}\right].
\label{eq:sig_prob}
\end{equation}

\section{Detectability}\label{sec:SNR}

The detectability of a burst is determined upon its SNR. We assume a detection threshold of $\rho = 10$. The SNR of an EMRB depends upon many parameters. For a given MBH, the most important is the periapse radius $r\sub{p}$. There is a good correlation between $\rho$ and $r\sub{p}$; other parameters specifying the inclination of the orbit or the orientation of the system with respect to the detector only produce scatter around this relation. The form of the $\rho$--$r\sub{p}$ relation depends upon the noise curve.

We parametrize the detectability in terms of a characteristic frequency
\begin{equation}
f_\ast = \sqrt{\frac{GM}{r\sub{p}^3}}.
\end{equation}
This allows comparison between different systems where the same periapse does not correspond to the same frequency, and thus the same point of the noise curve.

We also expect the SNR to scale with other quantities. Let us define a characteristic strain amplitude for a burst $h_0$; we expect $\rho \propto h_0$, where the proportionality will be set by a frequency-dependent function than includes the effect of the noise curve. Assuming that the strain is dominated by the quadrupole contribution (\citealt*[section 36.10]{Misner1973}; \citealt[section 17.9]{Hobson2006})
\begin{equation}
h_0 \sim \frac{G}{c^6}\frac{\mu}{R}\frac{\dd^2}{\dd t^2}\left(r^2\right),
\end{equation}
where $\mu$ is the CO mass, $R$ is the distance to the MBH, $t$ is time and $r$ is a proxy for the position of the orbitting object. The characteristic rate of change is set by $f_\ast$ and the characteristic length scale is set by $r\sub{p}$. Hence
\begin{align}
h_0 \sim {} & \frac{G}{c^6}\frac{\mu}{R}f_\ast^2 r\sub{p}^2 \\
 \sim {} & \frac{G^{5/2}}{c^6}\frac{\mu}{R}f_\ast^{-2/3}M^{2/3}.
\end{align}
Using this, we can factor out the most important dependencies to give a scaled SNR
\begin{equation}
\rho_\ast = \left(\frac{\mu}{M_\odot}\right)^{-1}\left(\frac{R}{\mathrm{Mpc}}\right)\left(\frac{M}{10^6 M_\odot}\right)^{-2/3}\rho.
\end{equation}

Space-based detectors are most sensitive from extreme-mass-ratio signals originating from MBHs with masses $10^5$--$10^6$. Higher mass objects produce signals at too low frequencies. We considered several nearby MBHs that were likely candidates for detectable burst signals. Details are given in \tabref{MBHs}.
\begin{table*}
%\begin{minipage}{\columnwidth}
 \centering
  \caption{Sample of nearby MBHs that are candidates for producing detectable EMRBs.\label{tab:MBHs}}
  \begin{tabular}{@{} l D{.}{.}{3.2} D{.}{.}{2.5} l @{}}
  \hline
   Galaxy & \multicolumn{1}{c}{$M/10^6 M_\odot$} & \multicolumn{1}{c}{$R/\mathrm{Mpc}$} & References \\
 \hline
 Milky Way (MW) & 4.31 & 0.00833& \citet{Gillessen2009} \\
 M32 & 2.5 & 0.770 & \\
 Andromeda (M31) & 140 & 0.770 &  \\
 Circinus & 1.1 & 2.82 & \citet{Graham2008,Greenhill2003,Karachentsev2007} \\
 NGC 4945 & 1.4 & 3.82	& \citet{Greenhill1997,Karachentsev2007} \\
 Sculptor (NGC 253) & 10 & 3.5 & \citet{Graham2011,Rodriguez-Rico2006,Rekola2005} \\
 NGC 3368 & 7.3 & 10.1 & \citet{Graham2011,Nowak2010,Tonry2001} \\
 NGC 3489 & 5.8 & 11.7 & \citet{Graham2011,Nowak2010,Tonry2001} \\
 NGC 4395 & 0.36 & 4.0& \citet{Peterson2005,Thim2004} \\
\hline
\end{tabular}
%\end{minipage}
\end{table*}
For each, we calculated SNRs at $\sim 10^4$ different periapse distances, uniformly distributed in log-space between the innermost orbit and $100 r\sub{g}$. Each had a spin and orbital inclination randomly chosen from distributions uniform in $a_\ast$ and $\cos \iota$.\footnote{The innermost orbit depends upon $a_ast$ and $\iota$, hence these are drawn first.} For every periapse, five SNRs were calculated, each having a different set of intrinsic parameters specifying the relative orientation of the MBH, the orbital phase and the position of the detector, drawn from appropriate uniform distributions. The scaled SNRs are plotted in \figref{scaled-SNR}. The plotted points are the average values of $\ln \rho_\ast$ calculated for each periapse distance.
\begin{figure}
\begin{center}
 %\includegraphics[width=0.43\textwidth]{Fig_SNR_ratio}
 \caption{Scaled signal-to-noise ratio for EMRBs as a function of characteristic frequency.\label{fig:scaled-SNR}}
   \end{center}
\end{figure}The curve shows that EMRB SNR does scale as expected, and $\rho_\ast$ can be describe as a one-parameter curve. There remains some scatter about this (removing the averaging over intrinsic parameters increases this to about an order of magnitude); however, it is good enough for rough calculations.

We approximate the trend with a best-fit curve
\begin{equation}
\rho_\ast = \alpha_1 f_\ast^{\beta_1} \left[1 + \left(\alpha_2 f_\ast\right)^{\beta_2}\right]\left[1 + \left(\alpha_3 f_\ast\right)^{\beta_4}\right]^{-\beta_5}.
\label{eq:scaled-SNR}
\end{equation}
To fit this, we treat the problem as if it were a likelihood maximisation, with each averaged point having a Gaussian likelihood with standard deviation defined from the scatter because of the variation in the intrinsic parameters. The optimised values are
\begin{equation}
\begin{split}
\alpha_1 = ; \quad \alpha_2 = ; \quad \alpha_3 = ;
\beta_1 = ; \quad \beta_2 = ; \quad \beta_3 = ; \quad \beta_4 = ;
\end{split}
\end{equation}

\section{Parameter inference}\label{sec:MCMC}

\section*{Acknowledgments}

The authors would like to thank Donald Lynden-Bell for useful conversations. CPLB is supported by STFC. JRG is supported by the Royal Society. The MCMC simulations were performed using the Darwin Supercomputer of the University of Cambridge High Performance Computing Service (\url{http://www.hpc.cam.ac.uk/}), provided by Dell Inc.\ using Strategic Research Infrastructure Funding from the Higher Education Funding Council for England.

\bibliographystyle{mn3e}
\bibliography{Extragalactic}

\bsp

\label{lastpage}

\end{document}
