\documentclass[useAMS,usedcolumn,usegraphicx,usenatbib]{mn2e}

%%%%% AUTHORS - PLACE YOUR OWN MACROS HERE %%%%%
\newcommand{\eqnref}[1]{(\ref{eq:#1})}
\newcommand{\figref}[1]{fig.~\ref{fig:#1}}
\newcommand{\Figref}[1]{Figure~\ref{fig:#1}}
\newcommand{\tabref}[1]{table~\ref{tab:#1}}
\newcommand{\secref}[1]{Sec.~\ref{sec:#1}}
\newcommand{\Secref}[1]{Section~\ref{sec:#1}}
\newcommand{\apref}[1]{Appendix~\ref{sec:#1}}

\newcommand{\units}[1]{\ensuremath{~\mathrm{#1}}}

\newcommand{\sub}[1]{\ensuremath{_\mathrm{#1}}}
\newcommand{\super}[1]{\ensuremath{^\mathrm{#1}}}
\newcommand{\dd}{\ensuremath{\mathrm{d}}}
\newcommand{\diff}[2]{\ensuremath{\frac{\dd {#1}}{\dd {#2}}}}
\newcommand{\partialdiff}[2]{\ensuremath{\frac{\partial {#1}}{\partial {#2}}}}
\newcommand{\intd}[4]{\ensuremath{\int_{#1}^{#2}{#3}\,\dd{#4}}}
\newcommand{\recip}[1]{\ensuremath{\frac{1}{#1}}}

\newcommand{\order}[1]{\ensuremath{\mathcal{O}({#1})}}
%\newcommand{\P}{\ensuremath{\mathrm{P}}}
%%%%%%%%%%%%%%%%%%%%%%%%%%%%%%%%%%%%%%%%%%%%%%%%

\title[Galactic Dynamics]{Galactic Dynamics}
\author[C. P. L. Berry and J. R. Gair]{C. P. L. Berry$^{1}$\thanks{E-mail:
cplb2@cam.ac.uk}  and J. R. Gair$^{1}$\\
$^{1}$Institute of Astronomy, University of Cambridge, Madingley Road, Cambridge, CB3 0HA}

\begin{document}

\date{\today}

\pagerange{\pageref{firstpage}--\pageref{lastpage}} \pubyear{2011}

\maketitle

\label{firstpage}

\begin{abstract}
I thought I would try out the MNRAS \LaTeX{} style.
\end{abstract}

\begin{keywords}
black hole physics -- celestial mechanics --  Galaxy: centre -- gravitational waves.
\end{keywords}

\section{Event rates}

\subsection{The distribution function}

We wish to calculate the probability that there is an encounter between a compact object on an orbital trajectory described by eccentricity $e$ and periapse radius $r_p$ and the supermassive black hole at the galactic centre. We begin by following the work of \citet{Bahcall1976, Bahcall1977} and assuming that the distribution function within the galactic core is just a function of the orbital energy; the number of stars is given by
\begin{equation}
N = \int \dd^3r \int \dd^3v f(\mathcal{E}).
\end{equation}
We will define the energy per unit mass of the orbit as
\begin{equation}
\mathcal{E} = \frac{v^2}{2} - \frac{GM_\bullet}{r}
\end{equation}
where $M_\bullet$ is the mass of the supermassive black hole. Close to the centre of the galactic core dynamics will be dominated by the influence of the black hole as it is significantly more massive than the surrounding stars. We will define the radius of influence for the black hole as
\begin{equation}
r_c = \frac{GM_\bullet}{\sigma_\star^2}
\end{equation}
where $\sigma^2$ is the line-of-sight velocity dispersion. We will assume that the mass of stars enclosed within the radius is greater than the black hole mass, which is much greater than the mass of a typical star $M_\star$~\citep{Bahcall1976}. We will define a reference number density from the enclosed mass as
\begin{equation}
m_\star(r_c) = \frac{4\pi r_c^3}{3}n_\star M_\star.
\end{equation}
Within the the core, the distribution function can be calculated using the approximation of Fokker-Planck formalism. The population of bound ($\mathcal{E} < 0$) stars is evolved numerically until a steady state is reached: the unbound ($\mathcal{E} > 0$) stars form a reservoir with an assumed Maxwellian distribution. Denoting a species of star by its mass $M$:
\begin{equation}
f_M(\mathcal{E}) = \frac{C_M n_\star}{(2\pi\sigma_M^2)^{3/2}} \exp\left(-\frac{\mathcal{E}}{\sigma_M^2}\right)
\label{eq:Unbound_DF}
\end{equation}
where $C_M$ is a normalisation constant.\footnote{$C_M$ determines the population ratio of species $M$ far from the black hole~\cite{Alexander2009}.} If different stellar species are in equipartition (as was assumed by \citet{Bahcall1976, Bahcall1977}) then we expect
\begin{equation}
M \sigma_M^2 = M_\star \sigma_\star^2.
\end{equation}
However, if the unbound stellar population has reached equilibrium by violent relaxation~\citep{Lynden-Bell1967}, then all mass groups are expected to have similar velocity dispersions: this has been used by \citet{Alexander2009, O'Leary2009} and will be assumed here. The steady-state distribution function is largely insensitive to this choice of boundary condition~\citep{Bahcall1977, Alexander2009}.

For bound orbits ($\mathcal{E} < 0$) the distribution function can be approximated as a power law
\begin{equation}
f_M(\mathcal{E}) = \frac{k_M n_\star}{(2\pi\sigma_M^2)^{3/2}}\left(-\frac{\mathcal{E}}{\sigma_M^2}\right)^{p_M}.
\end{equation}
The exponent $p_M$ varies depending upon the mass of the object, yielding mass segregation. For a system with a single mass component, \citet{Bahcall1976} find that $p = 1/4$. The normalisation constant $k_M$ reflects the relative abundances of the different species.\footnote{For a single mass population ($p = 1/4$) $k = 2 C$ gives a fit correct to within a factor a two~\citep{Bahcall1976,Keshet2009}, we will assume this holds for the dominant species of stars as although it will vary slightly with $p$ this change is small compared to errors introduced by fitting a simple power law~\citep{Hopman2006, Alexander2009}.}

\subsection{Model parameters}

We will be using the Fokker-Planck model of \citet{Hopman2006, Hopman2006a, Alexander2009}. This includes four stellar species: main sequence stars (MS), white dwarfs (WD), neutron stars (NS), and black holes (BH). Their properties are summarised in \tabref{HA}.
\begin{table}
\begin{minipage}{\columnwidth}
 \centering
  \caption{Stellar model parameters for the galactic centre using the results of \citet{Alexander2009} We use the main sequence star as our reference. The number fractions for unbound stars are estimates corresponding to a model of continuous star formation~\citep{Alexander2005}; \citet{O'Leary2009} arrive at the same proportions.\label{tab:HA}}
  \begin{tabular}{@{}lrrrr@{}}
  \hline
   Star & $M/M_\odot$ & $C_M/C_\star$ & $p_M$ & $k_M/k_\star$\footnote{\citet{Toonen2009}} \\
 \hline
 MS & $1.0$ & $1$ & $-0.1$ & $1$ \\
 WD & $0.6$ & $0.1$ & $-0.1$ & $0.09$ \\
 NS & $1.4$ & $0.01$ & $0$ & $0.01$  \\
 BH & $10$ & $0.001$ & $0.5$ & $0.008$ \\
\hline
\end{tabular}
\end{minipage}
\end{table}
The steeper power law for black holes means that they segregate about the massive black hole: they dominate in place of main sequence stars for radii $r < 10^{-4}r_c$.

We assume a black hole mass of $M_\bullet = (4.31 \pm 0.36) \times 10^6 M_\odot$~\cite{Gillessen2009} and a velocity dispersion of $\sigma = (103 \pm 20)\units{km\,s^{-1}}$~\cite{Tremaine2002}. This gives a core radius of $r_c = (1.75 \pm 0.70)\units{pc}$. Using the results of \citet{Ghez2008} we would expect the total mass of stars in the core to be $m_\star(r_c) = 6.4 \times 10^6 M_\odot$ (which is within $10\%$ of the value obtianed similarly from \citet{Genzel2003}). This gives a stellar density of $n_\star = 2.8 \times 10^5\units{pc^{-3}}$.

\subsection{Eccentricity \& periapsis}

We wish to make a change of variables to characterise orbits by their eccentricity $e$ and periapse radius $r_p$. The latter, unlike the semimajor axis, is always well defined regardless of eccentricity. For Keplerian orbits, the energy $\mathcal{E}$ and angular momentum $J$ per unit mass are entirely characterised by these parameters
\begin{eqnarray}
\mathcal{E} & = & -\frac{GM_\bullet(1 - e)}{2r_p},\\
J^2 & = & GM_\bullet(1 + e)r_p.
\end{eqnarray}
We start by decomposing the velocity into three orthogonal components: radial $v_r$, azimuthal $v_\phi$ and polar $v_\theta$. We will assume that the galactic core is spherically symmetric~\citep{Genzel2003, Schodel2007}, therefore we are only interested in the combination
\begin{equation}
v_\perp^2 = v_\phi^2 + v_\theta^2 = v^2 - v_r^2.
\end{equation}
Under this change of variables
\begin{equation}
\dd^2v = \dd v_r \dd v_\phi \dd v_\theta \rightarrow 2\pi v_\perp \dd v_r \dd v_\perp.
\end{equation}
The specific energy and angular momentum are given by
\begin{eqnarray}
\mathcal{E} & = & \frac{v_r^2 + v_\perp^2}{2} - \frac{GM_\bullet}{r},\\
J^2 & = & r^2 v_\perp^2.
\end{eqnarray}
If we combine these with our earlier expressions in terms of $e$ and $r_p$ we find
\begin{eqnarray}
v_\perp^2 & = & \frac{GM_\bullet(1 + e)r_p}{r^2},\\
v_r^2 & = & GM\left[\frac{2}{r} - \frac{(1 - e)}{r_p} - \frac{(1 + e)r_p}{r^2}\right].
\end{eqnarray}
From the later we can verify that the turning points of an orbit occur at
\begin{equation}
r = r_p,\quad \frac{1+e}{1-e}r_p;
\end{equation}
the periapse is the only turning point for orbits with $e > 1$. Since we now have expressions for $\{v_r, v_\perp\}$ in terms of $\{e, r_p\}$ we can calculate the Jacobian
\begin{eqnarray}
\left|\frac{\partial(v_r, v_\perp)}{\partial(e, r_p)}\right| & = & \partialdiff{v_r}{e}\partialdiff{v_\perp}{r_p} - \partialdiff{v_r}{r_p}\partialdiff{v_\perp}{e} \\
 & = & \recip{2v_rv_\perp}\frac{e}{r_p}\left(\frac{GM}{r}\right)^2.
\end{eqnarray}
Using this, we may rewrite our velocity element as
\begin{eqnarray}
\dd^3v & = & \dd v_r \dd v_\phi \dd v_\theta\\
 & = & 2\pi v_\perp \dd v_r \dd v_\perp\\
 & = & \frac{\pi e}{v_rr_p}\left(\frac{GM}{r}\right)^2\dd e \dd r_p.
\end{eqnarray}
As a consequence of our assumed spherical symmetry, the volume element can be expressed as
\begin{equation}
\dd^3r = 4\pi r^2 \dd r.
\end{equation}
Thus the phase space volume element can be expressed as
\begin{equation}
\dd^3r\dd^3v = \frac{4\pi^2(GM)^2e}{v_rr_p}\dd r\dd e \dd r_p.
\end{equation}
The number of stars in an element $\dd r\dd e \dd r_p$ is
\begin{equation}
n(r, e, r_p) = \frac{4\pi^2(GM)^2e}{v_rr_p}f(\mathcal{E}).
\end{equation}

From this, we can construct the expected number of stars to be on orbits with periapsis $r_p$ and eccentricity $e$. We will define this locally, allowing it to vary with position. The number of stars found in a small radius range $\delta r$ with given orbital properties should be given by the the total number of stars with these properties multiplied by the relative amount of time they spend in that range
\begin{equation}
n(r, e, r_p)\delta r = N(e, r_p; r)\frac{\delta t}{P(e, r_p)}
\end{equation}
where $N(e, r_p; r)$ is the total number of stars with orbits given by $\{e, r_p\}$ defined at $r$, $\delta t$ is the time spent in $\delta r$ and $P(e, r_p)$ is the period of the orbit. We will defer the definition of this time for unbound orbits for now. The time spent in the radius range is
\begin{equation}
\delta t = 2\frac{\delta r}{v_r},
\end{equation}
where the factor of $2$ is included to account for both inwards and outwards motion. Hence
\begin{eqnarray}
N(e, r_p; r) & = & \recip{2} v_r P n(r, e, r_p)\\
 & = & \frac{2\pi^2(GM)^2 P e}{r_p}f(\mathcal{E}).
\end{eqnarray}
The right hand side of this equation is independent of position, subject to the constraint that the radius is in the allowed range for the orbit $r_p \leq r \leq (1+e)r_p/(1-e)$, and so we may define $N(e, r_p) \equiv N(e, r_p; r)$. This is a consequence of the distribution function being dependent only upon an orbital constant of the motion.

If a burst of radiation is emitted each time a star passes through periapse, then the event rate for burst emission from orbits with parameters $\{e, r_p\}$, is given by
\begin{eqnarray}
\Gamma(e, r_p) & = & \frac{N(e, r_p)}{P(e, r_p)}\\
 & = & \frac{2\pi^2(GM)^2 e}{r_p}f(\mathcal{E}).
\label{eq:Gamma}
\end{eqnarray}
The orbital period drops out from the calculation, so we do not have to worry about an appropriate definition for unbound orbits.

From the event rate we may define a probability of seeing a given number of events subject to the assumption that these events are uncorrelated. The probability is simply given by the Poisson distribution; the probability of there being $r$ events is
\begin{equation}
\Pr(r|\Gamma(e, r_p)) = \frac{\Gamma^r\exp(-\Gamma)}{r!}.
\end{equation}
The probability of there being a burst from an orbit with periapse $r_p$ and eccentricity $e$ is hence
\begin{equation}
\Pr(r \neq 0|\Gamma(e, r_p)) = 1 - \Pr(r = 0|\Gamma(e, r_p)).
\end{equation}

To estimate the expectation of a quantity across all orbits we use
\begin{equation}
\left\langle X\right\rangle = \sum_r \int_0^\infty \dd e \int_0^\infty \dd r_p X(r;r_p,e)\Pr(r|\Gamma(e, r_p)).
\end{equation}
Since the probability decays rapidly for large $r$, we may truncate the sum to give the required level of accuracy,

To generate a representative sample for the orbital parameters $e$ and $r_p$, we use $\Gamma(e, r_p)$ as an unnormalised probability distribution and draw from it appropriately.

\subsection{The inner cut-off}

From \eqnref{Gamma} we see that the event rate is highly sensitive to the smallest value of the periapsis. The inner cut-off for $r_p$ could result from a number of different physical causes. Ultimately the orbits cannot encroach closer to the black hole than its last stable orbit. This depends upon the spin of the black hole, but is of the order of its Schwarzschild radius. Before we reach this point, however, there are other processes that may intervene to deplete the orbitting stars. Our treatment of these is approximate, however should hopefully produce reasonable estimates. We will consider three processes: tidal disruption, gravitational wave inspiral and collisional disruption. Tidal disruption imposes a definite definite cut-off, while the others use statistical arguments, they are therefore true for a typical star and it is unlikely that a star would be found beyond the imposed limits.

\subsubsection{Tidal disruption}

Tidal forces from the black hole can disrupt stars. This occurs at the tidal radius
\begin{equation}
r_t \simeq \left(\frac{M_\bullet}{M}\right)^{1/3}R_M
\end{equation}
where $R_M$ is the radius of the star~\citep{Kobayashi2004}. Any star on an orbit with $r_p < r_T$ will be disrupted in the cause of its orbit. Tidal disruption is most significant for MS stars since they are least dense. Calculated in this way, only MS stars and WDs would be tidally disrupted outside of the MBH's event horizon.

\subsection{Gravitational wave inspiral}

Stars orbitting about the black hole will continually radiate energy and angular momentum causing them to inspiral. Using the analysis of \citet{Peters1964} for Keplerian binaries, for bound orbits the orbit-averaged rate of change of the periapsis and eccentricity are
\begin{eqnarray}
\left\langle\diff{r_p}{t}\right\rangle & = & -\frac{64}{5}\frac{\Theta}{r_p^3}\frac{(1 - e)^{3/2}}{(1 + e)^{7/2}}\left(1 - \frac{7}{12}e + \frac{7}{8}e^2 + \frac{47}{192}e^3\right) \\
\left\langle\diff{e}{t}\right\rangle & = & -\frac{304}{5}\frac{\Theta}{r_p^4}\frac{e(1 - e)^{3/2}}{(1 + e)^{5/2}}\left(1 + \frac{121}{304}e^2\right),
\end{eqnarray}
where we have introduced
\begin{equation}
\Theta = \frac{G^3M_\bullet M(M_\bullet + M)}{c^5}.
\end{equation}
For a circular orbit the inspiral time form initial periapsis $r_{p0}$ is
\begin{equation}
\tau_c(r_{p0}) = \frac{5}{256}\frac{r_{p0}^4}{\Theta}.
\end{equation}
For an orbit of finite eccentricity ($0 < e < 1$), we can solve for the periapsis as a function of eccentricity
\begin{equation}
r_p(e) = A(1 + e)^{-1}\left(1 + \frac{121}{304}e^2\right)^{870/2299}e^{12/19},
\end{equation}
where $A$ is a constant fixed by the initial conditions: for an orbit with initial eccentricity $e_0$
\begin{equation}
A = (1 + e_0)\left(1 + \frac{121}{304}e_0^2\right)^{-870/2299}e_0^{-12/19}r_{p0}.
\end{equation}
The inspiral is complete when the eccentricity has decayed to zero. Consequently the inspiral time is~\citep{Peters1964}
\begin{equation}
\tau(r_{p0},e_0) = \intd{0}{e_0}{\frac{15}{304}\frac{A^4}{\Theta}\frac{e^{29/19}}{(1-e^2)^{3/2}}\left(1 + \frac{121}{304}e^2\right)^{1181/2299}}{e}.
\end{equation}
This is best evaluated numerically, however it may be written in closed from as
\begin{eqnarray}
\tau(r_{p0},e_0) & = & \tau_c(r_{p0})(1 + e_0)^4\left(1 + \frac{121}{304}e_0^2\right)^{-3480/2299} \nonumber\\*
 & & \times F_1\left(\frac{24}{19};\frac{3}{2},-\frac{1181}{2299};\frac{43}{19};e_0^2,-\frac{121}{304}e_0^2\right),
\end{eqnarray}
using the Appell hypergeometric function of the first kind $F_1(\alpha;\beta,\beta';\gamma;x,y)$ (\citealt{Olver2010}, 16.15.1).\footnote{For small eccentricities $\tau(r_{p0},e_0) \simeq \tau_c(r_{p0})[1 + 4e_0 + (273/43)e_0^2 + \order{e_0^3}]$.}

We will assume that an orbit is depleted of stars if the inspiral timescale is shorter than the relaxation timescale. This indicates the characteristic time for two-body collisions to change the velocity the star by order of itself~\citep{Binney1987}, and so indicates the time over which scattering may repopulate the orbit. \citet{Binney1987} derive the relaxation timescale from the diffusion coefficient of the Fokker-Planck equation (section 8.3.4), for a system with a purely Maxwellian distribution
\begin{equation}
\tau_R \simeq 0.34\frac{\sigma_\star^3}{GM_\star\rho_\star\ln\Lambda},
\end{equation}
where $\rho_\star$ is the mass density and the Coulomb logarithm is $\ln\Lambda = \ln(M_\bullet/M_\star)$. \citet{Bahcall1977} find a similar characteristic timescale, but with numerical prefactor $3/4\sqrt{8\pi} \simeq 0.15$. We shall adopt the former to be conservative. From this we estimate the relaxation time for an orbit as
\begin{eqnarray}
T_R(r_p,e) & = & \left(\frac{\langle v(r)\rangle\sub{orb}}{\sigma_\star}\right)^3\frac{M_\star\rho_\star}{\langle M(r)\rangle\sub{orb}\langle \rho(r)\rangle\sub{orb}}\tau_R\\
 & = & \left(\frac{2 E(e)}{\pi}\sqrt{\frac{r_c(1 - e)}{r_p}}\right)^{3}\frac{M_\star\rho_\star}{\langle M(r)\rangle\sub{orb}\langle \rho(r)\rangle\sub{orb}}\tau_R
\end{eqnarray}
using the velocity $\langle v(r)\rangle\sub{orb}$, stellar mass of encountered stars $\langle M(r)\rangle\sub{orb}$ and density $\langle \rho(r)\rangle\sub{orb}$ averaged over the duration of the orbit as characteristic quantities. The subscript $\mathrm{orb}$ indicates that we average along the trajectory of the orbit specified by parameters $\left\{r_p,e\right\}$.\footnote{In calculating $\langle \rho(r)\rangle\sub{orb}$ and $\langle M(r)\rangle\sub{orb}$ we include the tidal disruption of stars but not the possible depopulation of orbits due to inspiral or collisions. This would require solving iteratively for $n(r)$, which is not justifiable at this level of approximation. The relaxation timescale will be underestimated using this simple approach. The error will be greatest for low eccentricity orbits which spend most of their time in the same region of space.} While the former may be written in closed form using the complete elliptic integral of the second kind, the latter is best evaluated numerically.

Unbound stars only undergo a single periapse passage and only radiate one burst of radiation, we shall therefore neglect any evolution in their orbital parameters.\footnote{Using the analysis of \citet{Turner1977} it is possible to show that the change in eccentricity $\Delta e / e$ and periapsis $\Delta r_p / r_p$ for an extreme mass-ratio binary are less than $\order{\eta}\sqrt{e}$, where $\eta = M/M_\bullet$, and so are only important for very high eccentricity orbits (\apref{Unbound}). These are very high energy, and exponentially rare because of the Boltzmann factor in \eqnref{Unbound_DF}.}

\subsubsection{Collisions}

As a consequence of the higher densities in the galactic core, stars may undergo a large number of close encounters with other stars. It takes $20$--$30$ grazing collisions to disrupt a MS star~\citep{Freitag2006}. We shall ignore the possibility of disruption for the other species since they have much smaller cross-sectional areas. The number of collisions a star will undergo in a time interval $\delta t$ is
\begin{equation}
\delta K = n(r) \pi R_\star^2 v(r,e,r_p)\delta t.
\end{equation}
For circular orbits we can find the radius at which collisions will lead to disruptions by setting $\delta K = 30$ and $\delta t = \tau_R$. We use the relaxation timescale for the system as this is the time over which stars are replenished from the reservoir. For non-circular orbits we must consider variation with position. Using $\delta r = v_r \delta t$, and then converting to an integral, we have for bound orbits
\begin{equation}
K = 2\pi R_\star^2 \frac{\tau_R}{P(r_p,e)}\intd{r_p}{(1+e)r_p/(1-e)}{n(r)\frac{v(r,e,r_p)}{v_r(r,e,r_p)}}{r},
\end{equation}
where $P$ is the period of the orbit. Again we may set $K = 30$ to find the orbits for which stars will be disrupted within a relaxation timescale. For unbound stars we are only interested in stars that would become disrupted before their periapse passage, so
\begin{equation}
K = \pi R_\star^2 \intd{r_p}{r_c}{n(r)\frac{v(r,e,r_p)}{v_r(r,e,r_p)}}{r},
\end{equation}
assuming that the stars in the reservoir external to the core are unlikely to undergo close collisions.


\section*{Acknowledgments}


\bibliographystyle{mn2e}
\bibliography{Galactic}


\appendix

\section[]{Evolution of orbital parameters for unbound orbits due to gravitational wave emission}\label{sec:Unbound}

Following the approach of \citet{Turner1977} we can calculate the evolution of the eccentricity and periapse of a Keplerian binary through the loss of energy and angular momentum through carried away by gravitational radiation. The change in fractional eccentricity over an orbit, approximating the orbital parameters remain constant throughout the orbit, is
\begin{eqnarray}
\frac{\Delta e}{e} & = & -\frac{608}{15}\Sigma\left[\recip{(1+e)^{5/2}}\left(1 + \frac{121}{304}e^2\right)\cos^{-1}\left(-\recip{e}\right)\right. \nonumber\\*
 & & \left. + \frac{(e - 1)^{1/2}}{e^2(1+e)^2}\left(\frac{67}{456} + \frac{1069}{912}e^2 + \frac{3}{38}e^4\right)\right],
\end{eqnarray}
introducing dimensionless parameter
\begin{equation}
\Sigma = \frac{G^{5/2}M_\bullet M(M_\bullet+ M)}{c^5r_p^{5/2}}.
\end{equation}
Similarly, the fractional change in periapsis is
\begin{eqnarray}
\frac{\Delta r_p}{r_p} & = & -\frac{128}{5}\Sigma\left[\recip{(1+e)^{7/2}}\left(1 - \frac{7}{12}e + \frac{7}{8}e^2 + \frac{47}{192}e^3\right)\cos^{-1}\left(-\recip{e}\right)\right. \nonumber \\*
 & & \left. - \frac{(e - 1)^{1/2}}{e(1 + e)^3}\left(\frac{67}{288} - \frac{13}{8}e + \frac{133}{576}e^2 - \frac{1}{4}e^3 - \frac{1}{8}e^4\right)\right].
\end{eqnarray}
Both of these changes obtain their greatest magnitudes for large eccentricities, then
\begin{equation}
\frac{\Delta e}{e} \simeq \frac{\Delta r_p}{r_p} \simeq -\frac{16}{5}\Sigma e^{1/2}.
\end{equation}
For extreme mass-ratio binaries, as is the case here, the mass-ratio is a small quantity
\begin{equation}
\eta = M/M_\bullet \ll 1.
\end{equation}
The smallest possible periapsis is of order of the Schwarzschild radius of the massive black hole, such that 
\begin{equation}
r_p = \alpha\frac{GM_\bullet}{c^2}; \quad \alpha > 1.
\end{equation}
These give
\begin{equation}
\Sigma = \frac{\eta}{\alpha^{5/2}} < \eta \ll 1.
\end{equation}
Hence the changes in orbital parameters will be significant for
\begin{equation}
e \sim \frac{25}{256}\frac{\alpha^5}{\eta^2} > \frac{25}{256}\recip{\eta^2}.
\end{equation}

\bsp

\label{lastpage}

\end{document}




