\documentclass[useAMS,usedcolumn,usegraphicx,usenatbib]{mn2e}
\usepackage{amssymb,amstext,amsfonts} %% ... with default font
\usepackage[fleqn]{amsmath}
\usepackage{dcolumn}
\usepackage{hyperref}

%%%%% AUTHORS - PLACE YOUR OWN MACROS HERE %%%%%
\newcommand{\eqnref}[1]{(\ref{eq:#1})}
\newcommand{\figref}[1]{fig.~\ref{fig:#1}}
\newcommand{\Figref}[1]{Figure~\ref{fig:#1}}
\newcommand{\tabref}[1]{table~\ref{tab:#1}}
\newcommand{\secref}[1]{Sec.~\ref{sec:#1}}
\newcommand{\Secref}[1]{Section~\ref{sec:#1}}
\newcommand{\apref}[1]{Appendix~\ref{sec:#1}}

\DeclareMathOperator{\Ei}{Ei}
\DeclareMathOperator{\erf}{erf}
\DeclareMathOperator{\Beta}{B}

\newcommand{\units}[1]{\ensuremath{~\mathrm{#1}}}

\newcommand{\sub}[1]{\ensuremath{_\mathrm{#1}}}
\newcommand{\super}[1]{\ensuremath{^\mathrm{#1}}}
\newcommand{\dd}{\ensuremath{\mathrm{d}}}
\newcommand{\diff}[2]{\ensuremath{\frac{\dd {#1}}{\dd {#2}}}}
\newcommand{\partialdiff}[2]{\ensuremath{\frac{\partial {#1}}{\partial {#2}}}}
\newcommand{\intd}[4]{\ensuremath{\displaystyle \int_{#1}^{#2}{#3}\,\dd{#4}}}
\newcommand{\recip}[1]{\ensuremath{\dfrac{1}{#1}}}

\newcommand{\order}[1]{\ensuremath{\mathcal{O}({#1})}}
%\newcommand{\P}{\ensuremath{\mathrm{P}}}
%%%%%%%%%%%%%%%%%%%%%%%%%%%%%%%%%%%%%%%%%%%%%%%%

\title[Galactic Dynamics]{Galactic Dynamics}
\author[C. P. L. Berry and J. R. Gair]{C. P. L. Berry$^{1}$\thanks{E-mail:
cplb2@cam.ac.uk}  and J. R. Gair$^{1}$\\
$^{1}$Institute of Astronomy, University of Cambridge, Madingley Road, Cambridge, CB3 0HA}

\begin{document}

\date{\today}

\pagerange{\pageref{firstpage}--\pageref{lastpage}} \pubyear{2011}

\maketitle

\label{firstpage}

\begin{abstract}
I thought I would try out the MNRAS \LaTeX{} style.
\end{abstract}

\begin{keywords}
black hole physics -- celestial mechanics --  Galaxy: centre -- gravitational waves.
\end{keywords}

\section{Event rates}

\subsection{The distribution function}

We wish to calculate the probability that there is an encounter between a compact object on an orbital trajectory described by eccentricity $e$ and periapse radius $r\sub{p}$ and the massive black hole (MBH) at the galactic centre. To do so we must assume a particular distribution of stars. We begin by following the work of \citet{Bahcall1976, Bahcall1977} and assuming that the distribution function (DF) within the galactic core is just a function of the orbital energy; we define the energy per unit mass of the orbit as
\begin{equation}
\mathcal{E} = \frac{v^2}{2} - \frac{GM_\bullet}{r}
\end{equation}
where $M_\bullet$ is the mass of the MBH. The number of stars is given by
\begin{equation}
N = \int \dd^3r \int \dd^3v f(\mathcal{E}).
\end{equation}
Close to the centre of the galactic core dynamics will be dominated by the influence of the MBH as it is significantly more massive than the surrounding stars. We will define its radius of influence as~\citep{Frank1976}
\begin{equation}
r\sub{c} = \frac{GM_\bullet}{\sigma^2}
\label{eq:r_c}
\end{equation}
where $\sigma^2$ is the line-of-sight velocity dispersion. We will assume that the mass of stars enclosed within the radius is greater than the black hole mass, which is much greater than the mass of a typical star $M_\star$~\citep{Bahcall1976}. We will define a reference number density from the enclosed mass as
\begin{equation}
m_\star(r\sub{c}) = \frac{4\pi r\sub{c}^3}{3}n_\star M_\star.
\end{equation}
Within the core, the distribution function can be calculated using the approximation of Fokker-Planck formalism. The population of bound stars is evolved numerically until a steady state is reached: the unbound stars form a reservoir with an assumed Maxwellian distribution. Denoting a species of star by its mass $M$:
\begin{equation}
f_M(\mathcal{E}) = \frac{C_M n_\star}{(2\pi\sigma_M^2)^{3/2}} \exp\left(-\frac{\mathcal{E}}{\sigma_M^2}\right),\quad\mathcal{E} > 0,
\label{eq:Unbound_DF}
\end{equation}
where $C_M$ is a normalisation constant.\footnote{$C_M$ determines the population ratio of species $M$ far from the black hole~\citep{Alexander2009}.} If different stellar species are in equipartition (as was assumed by \citealt{Bahcall1976, Bahcall1977}) then we expect
\begin{equation}
M \sigma_M^2 = M_\star \sigma_\star^2.
\end{equation}
However, if the unbound stellar population has reached equilibrium by violent relaxation, then all mass groups are expected to have similar velocity dispersions:
\begin{equation}
\sigma_M = \sigma_\star = \sigma,
\end{equation}
and we have equipartition of energy per unit mass~\citep{Lynden-Bell1967}. This will be assumed here following \citet{Alexander2009, O'Leary2009}. The steady-state distribution function is largely insensitive to this choice~\citep{Bahcall1977, Alexander2009}.

For bound orbits the DF can be approximated as a power law~\citep{Peebles1972}
\begin{equation}
f_M(\mathcal{E}) = \frac{k_M n_\star}{(2\pi\sigma^2)^{3/2}}\left(-\frac{\mathcal{E}}{\sigma^2}\right)^{p_M},\quad\mathcal{E} < 0.
\label{eq:Bound_DF}
\end{equation}
The exponent $p_M$ varies depending upon the mass of the object, determining mass segregation. For a system with a single mass component, \citet{Bahcall1976} find that $p = 1/4$. The normalisation constant $k_M$ reflects the relative abundances of the different species.\footnote{For a single mass population ($p = 1/4$) $k = 2 C$ gives a fit correct to within a factor a two~\citep{Bahcall1976,Keshet2009}, we will assume this holds for the dominant species of stars as, although it will change slightly with $p$, variation is small compared to errors introduced by fitting a simple power law~\citep{Hopman2006, Alexander2009}.}

\subsection{Model parameters}

We will be using the Fokker-Planck model of \citet{Hopman2006, Hopman2006a, Alexander2009}. This includes four stellar species: main sequence stars (MS), white dwarfs (WD), neutron stars (NS), and black holes (BH). Their properties are summarised in \tabref{HA}. The behaviour of the Fokker-Planck model has been verified by $N$-body simulations~\citep{Preto2010}.
\begin{table}
\begin{minipage}{\columnwidth}
 \centering
  \caption{Stellar model parameters for the galactic centre using the results of \citet{Alexander2009} We use the main sequence star as our reference. The number fractions for unbound stars are estimates corresponding to a model of continuous star formation~\citep{Alexander2005}; \citet{O'Leary2009} arrive at the same proportions.\label{tab:HA}}
  \begin{tabular}{@{} l D{.}{.}{2.1} D{.}{.}{1.3} D{.}{.}{1.1} D{.}{.}{1.3} @{}}
  \hline
   Star & \multicolumn{1}{c}{$M/M_\odot$} & \multicolumn{1}{c}{$C_M/C_\star$} & \multicolumn{1}{c}{$p_M$} & \multicolumn{1}{c}{$k_M/k_\star$\footnote{\citet{Toonen2009}}} \\
 \hline
 MS & 1.0 & 1 & -0.1 & 1 \\
 WD & 0.6 & 0.1 & -0.1 & 0.09 \\
 NS & 1.4 & 0.01 & 0.0 & 0.01  \\
 BH & 10 & 0.001 & 0.5 & 0.008 \\
\hline
\end{tabular}
\end{minipage}
\end{table}
The steeper power law for black holes means that they segregate about the MBH: they dominate in place of main sequence stars for radii $r < 10^{-4}r\sub{c}$.

Binaries may form in the galactic centre, encouraged by its high stellar density~\citep{O'Leary2009}. However the binary fraction is still expected to be small~\citep{Hopman2009}. Consequently, and because binaries will be disrupted by the MBH for periapses smaller than
\begin{equation}
r\sub{B}  \simeq \left(\frac{M_\bullet}{M_1 + M_2}\right)^{1/3}a\sub{B},
\end{equation}
where $M_1$ and $M_2$ are the masses of the binary's components, and $a\sub{B}$ is the binary's semi-major axis [cf.\ \eqnref{Tidal} below], we shall ignore the possible presence of binaries.

We assume a black hole mass of $M_\bullet = (4.31 \pm 0.36) \times 10^6 M_\odot$~\citep{Gillessen2009} and a velocity dispersion of $\sigma = (103 \pm 20)\units{km\,s^{-1}}$~\citep{Tremaine2002}. This gives a core radius of $r\sub{c} = (1.7 \pm 0.7)\units{pc}$. Using the results of \citet{Ghez2008} we would expect the total mass of stars core to be $m_\star(r\sub{c}) = 6.4 \times 10^6 M_\odot$ (which is within $5\%$ of the value obtained similarly from \citealt{Genzel2003}). This gives a reference stellar density of $n_\star = 2.8 \times 10^5\units{pc^{-3}}$.

\subsection{Parameterising in terms of eccentricity \& periapsis}

We will characterise orbits by their eccentricity $e$ and periapse radius $r\sub{p}$. The latter, unlike the semimajor axis, is always well defined regardless of eccentricity. For Keplerian orbits, the energy $\mathcal{E}$ and angular momentum $J$ per unit mass are entirely characterised by these parameters
\begin{align}
\label{eq:Energy_ecc}
\mathcal{E} = {} & -\frac{GM_\bullet(1 - e)}{2r\sub{p}},\\
J^2 = {} & GM_\bullet(1 + e)r\sub{p}.
\end{align}
The distribution function, however, is defined per element of phase space: it is necessary to change variables from position and velocity to eccentricity and periapsis. We start by decomposing the velocity into three orthogonal components: radial $v_r$, azimuthal $v_\phi$ and polar $v_\theta$. We will assume that the galactic core is spherically symmetric~\citep{Genzel2003, Schodel2007}, therefore we are only interested in the combination
\begin{equation}
v_\perp^2 = v_\phi^2 + v_\theta^2 = v^2 - v_r^2.
\end{equation}
Under this change of variables
\begin{equation}
\dd^3v = \dd v_r \dd v_\phi \dd v_\theta \rightarrow 2\pi v_\perp \,\dd v_r \,\dd v_\perp.
\end{equation}
The specific energy and angular momentum are given by
\begin{align}
\mathcal{E} = {} & \frac{v_r^2 + v_\perp^2}{2} - \frac{GM_\bullet}{r},\\
J^2 = {} & r^2 v_\perp^2.
\end{align}
If we combine these with our earlier expressions in terms of $e$ and $r\sub{p}$ we find
\begin{align}
v_\perp^2 = {} & \frac{GM_\bullet(1 + e)r\sub{p}}{r^2}, \nonumber \\*
v_r^2 = {} & GM\left[\frac{2}{r} - \frac{(1 - e)}{r\sub{p}} - \frac{(1 + e)r\sub{p}}{r^2}\right].
\end{align}
From the latter we can verify that the turning points of an orbit occur at
\begin{equation}
r = r\sub{p}, \: \frac{1+e}{1-e}r\sub{p};
\end{equation}
the periapse is the only turning point for orbits with $e > 1$. Since we now have expressions for $\{v_r, v_\perp\}$ in terms of $\{e, r\sub{p}\}$ we can calculate the Jacobian
\begin{align}
\left|\frac{\partial(v_r, v_\perp)}{\partial(e, r\sub{p})}\right| = {} & \partialdiff{v_r}{e}\partialdiff{v_\perp}{r\sub{p}} - \partialdiff{v_r}{r\sub{p}}\partialdiff{v_\perp}{e} \\
 = {} & \recip{2v_rv_\perp}\frac{e}{r\sub{p}}\left(\frac{GM}{r}\right)^2.
\end{align}
Using this, we may rewrite our velocity element as
\begin{equation}
\dd^3v \rightarrow \frac{\pi e}{v_rr\sub{p}}\left(\frac{GM}{r}\right)^2\,\dd e \,\dd r\sub{p}.
\end{equation}
As a consequence of our assumed spherical symmetry, the volume element is
\begin{equation}
\dd^3r = 4\pi r^2 \,\dd r.
\end{equation}
Thus, the phase space volume element can be expressed as
\begin{equation}
\dd^3r\dd^3v = \frac{4\pi^2(GM)^2e}{v_rr\sub{p}}\,\dd r\,\dd e \,\dd r\sub{p}.
\end{equation}
The number of stars in an element $\dd r\,\dd e\,\dd r\sub{p}$ is
\begin{equation}
n(r, e, r\sub{p}) = \frac{4\pi^2(GM)^2e}{v_rr\sub{p}}f(\mathcal{E}).
\end{equation}

From this, we can construct the expected number of stars to be on orbits defined by $\{e, r\sub{p}\}$. We will define this locally, allowing it to vary with position. The number of stars found in a small radius range $\delta r$ with given orbital properties can be calculated by multiplying the total number of stars with these properties by the relative amount of time they spend in that range
\begin{equation}
n(r, e, r\sub{p})\delta r = N(e, r\sub{p}; r)\frac{\delta t}{P(e, r\sub{p})}
\end{equation}
where $N(e, r\sub{p}; r)$ is the total number of stars with orbits given by $\{e, r\sub{p}\}$ defined at $r$, $\delta t$ is the time spent in $\delta r$ and $P(e, r\sub{p})$ is the period of the orbit. We will defer the definition of this time for unbound orbits for now. The time spent in the radius range is
\begin{equation}
\delta t = 2\frac{\delta r}{v_r},
\end{equation}
where the factor of $2$ is included to account for both inwards and outwards motion. Hence
\begin{align}
N(e, r\sub{p}; r) = {} & \recip{2} v_r P n(r, e, r\sub{p})\\
 = {} & \frac{2\pi^2(GM)^2 e P}{r\sub{p}}f(\mathcal{E}).
\end{align}
The right hand side of this equation is independent of position, subject to the constraint that the radius is in the allowed range for the orbit $r\sub{p} \leq r \leq (1+e)r\sub{p}/(1-e)$, and so we may define $N(e, r\sub{p}) \equiv N(e, r\sub{p}; r)$. This is a consequence of the distribution function being dependent only upon a constant of the motion.

If a burst of radiation is emitted each time a star passes through periapse, then the event rate for burst emission from orbits with parameters $\{e, r\sub{p}\}$, is given by
\begin{align}
\Gamma(e, r\sub{p}) = {} & \frac{N(e, r\sub{p})}{P(e, r\sub{p})}\\
 = {} & \frac{2\pi^2(GM)^2 e}{r\sub{p}}f(\mathcal{E}).
\label{eq:Gamma}
\end{align}
The orbital period drops out from the calculation, so we do not have to worry about an appropriate definition for unbound orbits.

From the event rate we may define a probability of seeing a given number of events subject to the assumption that they are uncorrelated: it is given by the Poisson distribution. The probability of there being $r$ events is
\begin{equation}
\Pr(r|\Gamma(e, r\sub{p})) = \frac{\Gamma^r\exp(-\Gamma)}{r!}.
\end{equation}
The probability of there being a burst from an orbit with periapse $r\sub{p}$ and eccentricity $e$ is hence
\begin{equation}
\Pr(r \neq 0|\Gamma(e, r\sub{p})) = 1 - \Pr(r = 0|\Gamma(e, r\sub{p})).
\end{equation}

To estimate the expectation of a quantity across all orbits we use
\begin{equation}
\left\langle X\right\rangle = \sum\sub{R} \int_0^\infty \dd e \int_0^\infty \dd r\sub{p} X(r;r\sub{p},e)\Pr(r|\Gamma(e, r\sub{p})).
\end{equation}
Since the probability decays rapidly for large $r$, we may truncate the sum to give the required level of accuracy,

To generate a representative sample for the orbital parameters $e$ and $r\sub{p}$, we use $\Gamma(e, r\sub{p})$ as an unnormalised probability distribution and draw from it appropriately.

\subsection{The inner cut-off}

From \eqnref{Gamma} we see that the event rate is highly sensitive to the smallest value of the periapsis. The inner cut-off for $r\sub{p}$ could result from a number of different physical causes. Ultimately the orbits cannot encroach closer to the black hole than its last stable orbit. This depends upon the spin of the black hole, but is of the order of its Schwarzschild radius. Before we reach this point, however, there are other processes that may intervene to deplete the orbiting stars. Our treatment of these is approximate, but should hopefully produce reasonable estimates. We will consider three processes: tidal disruption, gravitational wave inspiral and collisional disruption. Tidal disruption imposes a definite cut-off, while the others use statistical arguments; they are therefore true for a typical star, and it is unlikely that a star would be found beyond the imposed limits. For these methods we will need to define a reference time-scale for relaxation, over which two-body interactions lead to significant changes in the orbital properties of the star. This is done in \secref{Relax}, with further details found in \apref{time-scale}.

\subsubsection{Tidal disruption}

Tidal forces from the black hole can disrupt stars. This occurs at the tidal radius
\begin{equation}
r\sub{T} \simeq \left(\frac{M_\bullet}{M}\right)^{1/3}R_M
\label{eq:Tidal}
\end{equation}
where $R_M$ is the radius of the star~\citep{Hills1975, Rees1988, Kobayashi2004}. Any star on an orbit with $r\sub{p} < r\sub{T}$ will be disrupted in the course of its orbit. Parameterising orbits by their periapsis allows us to easily determine which stars should be disrupted. We do not include the full effects of the loss cone~\citep{Frank1976, Lightman1977} as these were not incorporated into the Fokker-Planck calculations~\citep{Hopman2009}.\footnote{The loss cone includes a region in angular momentum space that is depleted of stars because two-body scattering would allow diffusion over an orbital period to an orbit subject to tidal disruption.} The effect of the loss cone should be small, only modifying the DF by a logarithmic term~\citep{Lightman1977, Bahcall1977}. Its effects are diluted by resonant relaxation~\citep{Toonen2009}.

Tidal disruption is most significant for MS stars since they are least dense; calculated in this way, only MS stars would be tidally disrupted outside of the MBH's event horizon~\citep{Sigurdsson1997}. The tidal radius defines the cut-off for periapsis of high eccentricity ($e \ga 1$) orbits~\citep{Lightman1977}.

\subsubsection{Relaxation time-scale}\label{sec:Relax}

The motion of a star is determined not only by the dominant influence of the central MBH, but also by the other stars. The potential of the stars may be split into two components: the smooth background potential representing the average distribution of stars, and statistical fluctuations from random deviations in the stellar distribution. The former only contributes to the stars' orbits: we neglect this since we are more interested in the influence of the MBH. The latter may be approximated as a series of two-body encounters. These lead to scattering, in a manner much like Brownian motion~\citep{Bekenstein1992,Moaz1993,Nelson1999}.

The two-body interactions mostly lead to small deflections. Over time, these may accumulate into a significant change in the dynamics. The relaxation time-scale characterises the time taken for this to happen~\citep{Binney1987}. It therefore quantifies the time over which an orbit may be repopulated by scattering. There are a variety of definitions for the relaxation time-scale. For a system with a purely Maxwellian distribution, the time-scale has form
\begin{equation}
\tau\sub{R} \simeq \kappa\frac{\sigma^3}{GM_\star\rho_\star\ln\Lambda},
\end{equation}
where $\rho_\star$ is the mass density, the Coulomb logarithm is $\ln\Lambda = \ln(M_\bullet/M_\star)$, and $\kappa$ is a dimensionless number. In his pioneering work, \citet{Chandrasekhar1941, Chandrasekhar1960} defined the time-scale as the period over which the squared change in energy was equal to the kinetic energy squared, this gives $\kappa = 9/16\sqrt{\pi} \simeq 0.32$. Subsequently, \citet{Chandrasekhar1941a} described relaxation statistically, describing fluctuations in the gravitational field probabilistically, this gives $\kappa = 9/2(2\pi)^{3/2} \simeq 0.29$. \citet{Bahcall1977} define a reference time-scale from their Boltzmann equation with $\kappa = 3/4\sqrt{8\pi} \simeq 0.15$; this is equal to the reference time-scale defined as the reciprocal of the coefficient of dynamical friction by \citet{Chandrasekhar1943a, Chandrasekhar1943}. \citet{Spitzer1958} define a reference time-scale from the gravitational Boltzmann equation of \citet{Spitzer1951} where $\kappa = \sqrt{2}/\pi \simeq 0.45$. Following \citet{Spitzer1971}, \citet[section 8.3.4]{Binney1987} estimate the time-scale from the velocity diffusion coefficient of the Fokker-Planck equation yielding $\kappa \simeq 0.34$.

All these approaches yield consistent values, suggesting, as a first approximation, any approach should be valid. We will follow the classic treatment of \citet[chapter 2]{Chandrasekhar1960} which is transparent in its assumptions, adapting from a Maxwellian distribution of velocities to one derived from the DFs \eqnref{Unbound_DF} and \eqnref{Bound_DF}. Since there is uncertainty in the astrophysical parameters, we will not be concerned by small discrepencies in the numerical prefactor that result from the simplifying approximations of this approach.

\subsubsection{Gravitational wave inspiral}

Stars orbiting about the black hole will continually emit gravitational radiation, this carries away energy and angular momentum, causing the stars to inspiral. Using the analysis of \citet{Peters1963,Peters1964} for Keplerian binaries it is possible to define a characteristic inspiral time-scale from the rate of change of energy: we will assume that an orbit is depleted of stars if the characteristic inspiral time-scale is shorter than the relaxation time-scale. For consistency with the definition of the relaxation time-scale, we will define the characteristic inspiral timescale as
\begin{equation}
\tau\sub{GW} \simeq \mathcal{E}\langle\diff{\mathcal{E}}{T}\rangle,
\end{equation}
where the term in angular brackets is the orbit-averaged rate of energy radiation. Using \eqnref{Energy_ecc} and \citet{Peters1963}
\begin{align}
\tau\sub{GW} \simeq {} & \frac{5}{64}\frac{c^5r\sub{p}}{G^3MM_\bullet\left(M + M\bullet\right)}\frac{(1+e)^{7/2}}{(1-e)^{1/2}}\left(1+\frac{73}{24}e^2 + \frac{37}{96}e^4\right)^{-1} \\
 \approx {} & \frac{5}{64}\frac{c^5r\sub{p}}{G^3MM_\bullet^2}\frac{(1+e)^{7/2}}{(1-e)^{1/2}}\left(1+\frac{73}{24}e^2 + \frac{37}{96}e^4\right)^{-1}.
\end{align}
For comparison, the total time taken for the inspiral is given in \eqnref{Bound_inspiral}. The characteristic time-scale is a better measure of the depletion of an orbit as it only depends upon the parameters of that orbit, and not its future evolution. When comparing with the relaxation timescale we are in effect comparing rates of change, with the shorter timescale highlighting the more rapid process that will dominate the evolution.

Unbound stars only undergo a single periapse passage and only radiate one burst of radiation; we shall therefore neglect any evolution in their orbital parameters.\footnote{Using the analysis of \citet{Turner1977} it is possible to show that the relative changes in eccentricity $\Delta e / e$ and periapsis $\Delta r\sub{p} / r\sub{p}$ for an extreme mass-ratio binary are less than $\sqrt{e}\,\order{\eta}$, where $\eta = M/M_\bullet$ is small, and so are only important for very high eccentricity orbits (\apref{Unbound}). These are very high energy, and exponentially rare because of the Boltzmann factor in \eqnref{Unbound_DF}.}

The $(1-e)^{-1/2}$ dependence of $\tau\sub{GW}$ for bound orbits smoothly connects the two regimes. The rate of change of energy goes to zero as a consequence of the assumption that the orbital parameters do not change over the cause of a single orbit. This was necessary to calculate the orbit-averaged quantity. This is a valid approximation since the large mass-ratio ensures a slow evolution of the system (\apref{Unbound}).

\subsubsection{Collisions}

As a consequence of the higher densities in the galactic core, stars may undergo a large number of close encounters with other stars. A star is unlikely to be disrupted by a single collision~\citep{Freitag2005}, rather it takes $20$--$30$ grazing collisions to disrupt a MS star~\citep{Freitag2006}. We shall ignore the possibility of disruption for the other species since they are harder to pull apart: as a consequence of smaller cross-sectional areas, none of the other species would undergo 30 collisions with a periapsis greater than the cut-off implied by gravitational wave inspiral for bound orbits, or the MBH's event horizon for unbound orbits. The number of collisions a star will undergo in a time interval $\delta t$ is
\begin{equation}
\delta K = n(r) A v(r,e,r\sub{p})\delta t,
\end{equation}
where $A$ is the star's cross-sectional area for a grazing collision. We shall assume that the relative velocity of the colliding stars is much greater than the escape velocity of the star so we may neglect the effects of gravitational focusing, then the cross-sectional area is simply the geometric $A = \pi R_\star^2$.

For circular orbits we can find the radius at which collisions will lead to disruptions by setting $\delta K = 30$ and $\delta t = \tau\sub{R}$. We use the relaxation time-scale for the system as this is the time over which stars are replenished from the reservoir. For non-circular orbits we must consider variation with position. Using $\delta r = v_r \delta t$, and then converting to an integral, we have for bound orbits
\begin{equation}
K = 2\pi R_\star^2 \frac{\tau\sub{R}}{P(r\sub{p},e)}\intd{r\sub{p}}{(1+e)r\sub{p}/(1-e)}{n(r)\frac{v(r,e,r\sub{p})}{v_r(r,e,r\sub{p})}}{r},
\end{equation}
where $P$ is the period of the orbit. Again we may set $K = 30$ to find the orbits for which stars will be disrupted within a relaxation time-scale. For unbound stars we are only interested in stars that would become disrupted before their periapse passage, so
\begin{equation}
K = \pi R_\star^2 \intd{r\sub{p}}{r\sub{c}}{n(r)\frac{v(r,e,r\sub{p})}{v_r(r,e,r\sub{p})}}{r},
\end{equation}
assuming that the stars in the reservoir external to the core are unlikely to undergo close collisions.


\section*{Acknowledgments}
CPLB is supported by STFC. JRG is supported by the Royal Society.

\bibliographystyle{mn2e}
\bibliography{Galactic}

\appendix

\begin{onecolumn}

\section{Chandrasekhar's relaxation time-scale}\label{sec:time-scale}

\citet[chapter 2]{Chandrasekhar1960} defined a relaxation time-scale for a stellar system by approximating the fluctuations in the stellar gravitational potential as a series of two-body encounters. The time over which the squared change in energy is equal to the (initial) kinetic energy of the star is the time taken for relaxation. This relaxation is explained through dynamical friction~\citep{Chandrasekhar1943a, Binney1987}. This can be understood intuitively as the drag induced on a star by the overdensity of field stars deflected by its passage~\citep{Mulder1983}. In the interaction between the star and its gravitational wake, energy and momentum are exchanged, accelerating some stars, decelerating others.

Chandrasekhar's approach has proved exceedingly successful despite the number of simplifying assumptions inherent in the model which are not strictly applicable to systems such as the Galactic centre. We will not attempt to fix these deficiencies, rather the only modification to the standard picture we shall include will be to substitute the velocity distribution: while the canonical formulation uses a simple homogeneous Gaussian distribution, we shall use a distribution derived from the distribution functions \eqnref{Unbound_DF} and \eqnref{Bound_DF}.

Other authors have built upon the work of Chandrasekhar by considering inhomogeneous stellar distributions, via perturbation theory~\citep{Lynden-Bell1972,Tremaine1984,Weinberg1986}. Further extensions have been produced using the tools of linear response theory and the fluctuation-dissipation theory~\citep[chapter 7]{Landau1958}, which allows relaxation of certain assumptions, such as homogeneity~\citep{Bekenstein1992,Moaz1993,Nelson1999}. We will not attempt to employ such sophisticated techniques at this stage.

\subsection{Chandrasekhar's derivation of the change in energy}

We will consider the interaction of a field star, denoted by 1, with a test star, 2; the centre of gravity and relative velocities are
\begin{subequations}
\begin{align}
\boldsymbol{V}\sub{g} =  {} & \recip{m_1 + m_2}\left(m_1 \boldsymbol{v}_1 + m_2 \boldsymbol{v}_2\right);\\
\boldsymbol{V} =  {} & \boldsymbol{v}_1 - \boldsymbol{v}_2.
\end{align}
\label{eq:Vs}
\end{subequations}
Hence
\begin{subequations}
\begin{align}
v_1^2 = {} & V\sub{g}^2 - 2\frac{m_2}{m_1 + m_2}V\sub{g}V \cos\Phi + \left(\frac{m_2}{m_1 + m_2}\right)^2V^2;\\
v_2^2 = {} & V\sub{g}^2 + 2\frac{m_1}{m_1 + m_2}V\sub{g}V \cos\Phi + \left(\frac{m_1}{m_1 + m_2}\right)^2V^2,
\end{align}
\end{subequations}
where $\Phi$ is the angle between $\boldsymbol{V}\sub{g}$ and $\boldsymbol{V}$, and
\begin{subequations}
\begin{align}
V\sub{g}^2 = {} & \recip{(m_1 + m_2)^2}\left(m_1v_1^2 + m_2v_2^2 + 2 m_1 m_2 v_1 v_2 \cos\theta\right);\\
V^2 = {} & v_1^2 + v_2 - 2 v_1 v_2 \cos\theta,
\end{align}
\label{eq:V2s}
\end{subequations}
where $\theta$ is the angle between $\boldsymbol{v}_1$ and $\boldsymbol{v}_2$. The change in energy of during the interaction is
\begin{align}
\Delta E = {} & \recip{2} m_2 \left({v'_2}^2 - v_2^2\right)\\
 = {} & \frac{m_1 m_2}{m_1 + m_2}V\sub{g}V\left(\cos\Phi' - \cos\Phi\right),
\end{align}
using a primed variables for values after the interaction, and unprimed ones for before. If we project the angle out onto the orbital plane
\begin{equation}
\Delta E = \frac{m_1 m_2}{m_1 + m_2}V\sub{g}V\left(\cos\phi' - \cos\phi\right)\cos i,
\end{equation}
where $\phi$ is the angle in the plane, and $i$ is the inclination of $\boldsymbol{V}\sub{g}$ out of the plane. We define the deflection angle $\psi$ such that
\begin{equation}
\phi' - \phi = \pi - 2\psi,
\end{equation}
hence
\begin{equation}
\Delta E = -2\frac{m_1 m_2}{m_1 + m_2}V\sub{g}V\cos(\phi - \psi)\cos\psi\cos i.
\end{equation}

We now need to calculate the encounter rate. This requires us to know number of field stars per unit volume and per volume space element. We will make the simplifying assumption that the density of stars is uniform. This is not the case for the Galactic centre; however, we will approximate it as so in order to make the problem tractable. It may seem especially bad for treatment of compact objects very close to the central MBH, inside the radius where main sequence stars would be tidally disrupted; however, it must be remembered that it is small angle deflections, rather than large deflections from close collisions, that are most important for relaxation. Using an averaged value, the number of stars is
\begin{equation}
\dd N = n(v_1, \theta, \phi)\,\dd v_1 \,\dd \theta \,\dd \varphi \,\dd^3r,
\end{equation}
using spherical polar coordinates for velocity space. Using $D$ as the impact parameter for the encounter and $\Theta$ for the angle between the fundamental plane (containing $\boldsymbol{v}_1$ ans $\boldsymbol{v}_2$) and the the orbital plane, the number of events in time interval $\delta t$ is
\begin{equation}
\dd \Gamma =  n(v_1, \theta, \varphi)\,\dd v_1 \,\dd \theta \,\dd \varphi \frac{\dd \Theta}{2\pi} 2\pi D \,\dd D V \,\delta t.
\end{equation}
The squared change in energy for these encounters is
\begin{align}
\Delta E^2(v_1,\theta,\varphi,D,\Theta) = {} & \left(\Delta E\right)^2\,\dd \Gamma\\
 = {} & 4 n(v_1,\theta,\varphi)V\sub{g}^2V^3\left(\frac{m_1m_2}{m_1+m_2}\right)^2\cos^2i\cos^2(\phi-\psi)\cos^2\psi D\,\dd v_1\,\dd\theta\,\dd\varphi\,\dd\Theta\,\dd D\,\delta t.
\end{align}
We must integrate out all these dependencies.

Since we have assumed that the stellar density does not depend upon position, we can simply integrate over the impact parameter; this is related to the deflection angle by
\begin{equation}
\recip{\cos^2\psi} = 1 + \frac{D^2V^4}{G(m_1 + m_2)}.
\end{equation}
Thus
\begin{equation}
\Delta E^2(v_1,\theta,\varphi,\psi,\Theta) =  4 n(v_1,\theta,\varphi)\frac{V\sub{g}^2}{V}G^2m_1^2 m_2^2\cos^2i\frac{\cos^2(\phi-\psi)\sin\psi}{\cos\psi} \,\dd \psi\,\dd v_1\,\dd\theta\,\dd\varphi\,\dd\Theta\,\delta t.
\end{equation}
The integral over $\psi$ is
\begin{align}
I(\psi_0) = {} & \intd{0}{\psi_0}{\frac{\cos^2(\phi-\psi)\sin\psi}{\cos\psi}}{\psi}\\
 = {} & \frac{\sin 2\phi}{2}\left(\psi_0 - \frac{\sin 2\psi_0}{2}\right) - \frac{\cos 2\phi}{2}\left(\frac{\cos 2\psi_0}{2}\right) - \sin^2\phi\ln(\cos\psi_0).
\end{align}
Naively we would think that the upper limit for the deflection limit should be $\psi_0 = \pi/2$; however, this would introduce a logarithmic divergence. In actuality there will be a physical cut-off, reflecting a finite bound for the maximum impact parameter $D_0$~\citep{Weinberg1986}. This will be set by the scale of the system, beyond which scattering is negligible. While the logarithmic term is finite, it will still be large, dominating the other terms which are $\order{1}$; we shall therefore neglect the subdominant terms
\begin{equation}
\Delta E^2(v_1,\theta,\varphi,\Theta) \simeq  4 n(v_1,\theta,\varphi)\frac{V\sub{g}^2}{V}G^2m_1^2 m_2^2\cos^2i\sin^2\phi\ln\left(\recip{\cos\psi_0}\right)\,\dd v_1\,\dd\theta\,\dd\varphi\,\dd\Theta\,\delta t.
\end{equation}

Next, we integrate over the orbital plane inclination using
\begin{equation}
\cos i\sin\phi = \sin\Phi\cos\Theta,
\end{equation}
so that
\begin{equation}
\Delta E^2(v_1,\theta,\varphi) \simeq  4\pi n(v_1,\theta,\varphi)\frac{V\sub{g}^2}{V}G^2m_1^2 m_2^2\sin^2\Phi\ln\left(\recip{\cos\psi_0}\right)\,\dd v_1\,\dd\theta\,\dd\varphi\,\delta t.
\end{equation}
We are now left with just the velocity variables.

An expression for $\sin^2\Phi$ can be obtained from \eqnref{Vs} and \eqnref{V2s}, after some rearrangement
\begin{equation}
\frac{V_g^2}{V}\sin^2\Phi = \frac{v_1^2v_2^2\sin^2\theta}{\left(v_1^2 + v_2^2 - 2v_1 v_2 \cos\theta\right)^{3/2}}.
\end{equation}
To proceed further we must specify the form of $n(v_1,\theta,\varphi)$, if we assume isotropy
\begin{equation}
n(v_1,\theta,\varphi) = n(v_1)\recip{4\pi}\sin\theta.
\end{equation}
The integral over $\varphi$ is then trivial,
\begin{equation}
\Delta E^2(v_1,\theta) \simeq \pi n(v_1)\frac{G^2m_1^2 m_2^2v_1^2v_2^2\sin^3\theta}{\left(v_1^2 + v_2^2 - 2v_1 v_2 \cos\theta\right)^{3/2}}\ln\left[1 + \frac{D_0\left(v_1^2 + v_2^2 - 2v_1 v_2 \cos\theta\right)^2}{G^2\left(m_1 + m_2\right)^2}\right]\,\dd v_1\,\dd\theta\,\delta t.
\end{equation}

To integrate over $\theta$ it is easier to recast in terms of $V$; the integral is
\begin{align}
J = {} & v_1^2 v_2^2 \intd{0}{\pi}{\frac{\sin^3\theta}{\left(v_1^2 + v_2^2 - 2v_1 v_2 \cos\theta\right)^{3/2}}\ln\left[1 + \frac{D_0\left(v_1^2 + v_2^2 - 2v_1 v_2 \cos\theta\right)^2}{G^2\left(m_1 + m_2\right)^2}\right]}{\theta} \\
 = {} & v_1 v_2 \intd{V_-}{V_+}{\frac{\sin^2\theta}{V^2}\ln\left(1 + q^2V^4\right)}{V},
\end{align}
where the limits are
\begin{equation}
V_+ = v_1 + v_2; \quad V_- = |v_1 - v_2|,
\end{equation}
and we have introduced
\begin{equation}
q = \frac{D_0}{G\left(m_1+m_2\right)}.
\end{equation}
Using \eqnref{V2s} to rearrange, and then integrating by parts gives
\begin{align}
J = {} & \recip{4 v_1 v_2} \intd{V_-}{V_+}{\frac{\left(V_+^2 - V^2\right)\left(V^2 - V_-^2\right)}{V^2}\ln\left(1 + q^2V^4\right)}{V} \\
 = {} & \recip{4 v_1 v_2} \left\{\left[\frac{3V_+^2V_-^2 + \left(V_+^2 + V_-^2\right)V^2 - V^4}{3V}\right]^{V_+}_{V_-} - \intd{V_-}{V_+}{\frac{3V_+^2V_-^2 + \left(V_+^2 + V_-^2\right)V^2 - V^4}{3V}\frac{4q^2V^3}{1+ q^2V^4}}{V}\right\};
\end{align}
the former piece still contains the logarithmic term which we know must be large. It is therefore the dominant piece of the integral and we neglect the latter~\citep{Chandrasekhar1941},
\begin{equation}
J \simeq \recip{6 v_1 v_2} \left[\left(3V_-^2 + V_+^2\right)V_+\ln\left(1 + q^2V_+^4\right) - \left(3V_+^2 + V_-^2\right)V_-\ln\left(1 + q^2V_-^4\right)\right].
\end{equation}
This may be further simplified, reusing the limit of large $q$~\citep{Chandrasekhar1941,Chandrasekhar1941b},
\begin{align}
J \simeq {} & \recip{3 v_1 v_2} \left[\left(3V_-^2 + V_+^2\right)V_+\ln\left(qV_+^2\right) - \left(3V_+^2 + V_-^2\right)V_-\ln\left(qV_-^2\right)\right] \\
 \simeq {} & \frac{4}{3v_1v_2}\begin{cases}
\left(v_1^3 + v_2^3\right)\ln\left[q\left(v_1 + v_2\right)^2\right] - \left(v_2^3 - v_1^3\right)\ln\left[q\left(v_1 - v_2\right)^2\right] & v_1 \leq v_2 \\
\left(v_1^3 + v_2^3\right)\ln\left[q\left(v_1 + v_2\right)^2\right] - \left(v_1^3 - v_2^3\right)\ln\left[q\left(v_1 - v_2\right)^2\right] & v_1 \geq v_2
\end{cases} \\
 \simeq {} & \frac{8}{3}\begin{cases}
\dfrac{v_2^2}{v_1}\ln\left(\dfrac{v_1 + v_2}{v_2 - v_1}\right) + \dfrac{v_1^2}{v_2}\left[\ln\left(qv_2^2\right) + \ln\left(1 - \dfrac{v_1^2}{v_2^2}\right)\right] & v_1 < v_2 \\
v_2\left[\ln\left(qv_2^2\right) + \ln 4\right] & v_1 = v_2\\
\dfrac{v_1^2}{v_2}\ln\left(\dfrac{v_1 + v_2}{v_1 - v_2}\right) + \dfrac{v_2^2}{v_1}\left[\ln\left(qv_2^2\right) + \ln\left(\dfrac{v_1^2}{v_2^2} - 1\right)\right] & v_1 > v_2
\end{cases} \\
\approx {} & \frac{8}{3}\begin{cases}
\dfrac{v_1^2}{v_2}\ln\left(qv_2^2\right) & v_1 \leq v_2 \\
\dfrac{v_2^2}{v_1}\ln\left(qv_2^2\right) & v_1 \geq v_2
\end{cases}.
\end{align}
This form maintains the correct limit for $v_2 \rightarrow 0$. We are left with
\begin{equation}
\Delta E^2(v_1) \approx \frac{8\pi}{3} n(v_1)G^2m_1^2 m_2^2\ln\left(qv_2^2\right)\left\{\begin{array}{ll}\dfrac{v_1^2}{v_2} & v_1 \leq v_2\\ \dfrac{v_2^2}{v_1} & v_1 \geq v_2 \end{array}\right\}\,\dd v_1\delta t.
\end{equation}
The final integral requires a specific form for the velocity distribution.

\subsection{Velocity distributions}

The velocity space distribution function can be obtained by integrating out the spatial dependence in the full DF
\begin{equation}
f(v) = \int \dd^3r f(\mathcal{E}).
\end{equation}
As we are restricting our attention to the Galactic centre and assuming spherical symmetry
\begin{equation}
f(v) = 4\pi\intd{0}{r\sub{c}}{r^2f(\mathcal{E})}{r},
\end{equation}
where $r_c$ is defined by \eqnref{r_c}. It is useful to work in terms of dimensionless variables
\begin{align}
x = {} & \frac{\mathcal{E}}{\sigma^2}; \\
w = {} & \frac{v^2}{2\sigma^2}.
\end{align}
Changing the integral to be over dimensionless energy
\begin{equation}
f(v) = 4\pi r\sub{c}^3\intd{\infty}{w - 1}{\frac{f(x)}{(w-x)^4}}{x},
\end{equation}
keeping $w$ as a function of $v$.

The DF for unbound stars is assumed to be Maxwellian; it is defined in \eqnref{Unbound_DF} and is applicable for $x > 0$, implying $w > 1$:
\begin{align}
f_{\mathrm{u},\,M}(v) = {} & \frac{N_\ast}{\left(2\pi\sigma^2\right)^{3/2}}C_M \intd{0}{w-1}{\frac{\exp(-x)}{(w-x)^4}}{x} \\
 = {} & \frac{N_\ast}{\left(2\pi\sigma^2\right)^{3/2}}C_M\epsilon\left(\frac{v^2}{2\sigma^2}\right),
\end{align}
introducing
\begin{equation}
\epsilon(w) = \recip{2}\left\{\exp(-w)\left[4\exp(1) + \Ei(w) - \Ei(1)\right] - \frac{2 + w + w^2}{w^3}\right\},
\end{equation}
where $\Ei(x)$ is the exponential integral.

The DF for bound stars is approximated as a simple power law as defined in \eqnref{Bound_DF}, it is applicable for $x < 0$:
\begin{equation}
f_{\mathrm{b},\,M}(v) = \frac{N_\ast}{\left(2\pi\sigma^2\right)^{3/2}}k_M\begin{cases}
\intd{-\infty}{w-1}{\dfrac{(-x)^{p_M}}{(w-x)^4}}{x} & w \leq 1 \\
\intd{-\infty}{0}{\dfrac{(-x)^{p_M}}{(w-x)^4}}{x} & w \geq 1
\end{cases}.
\end{equation}
The integral may be evaluated in terms of the hypergeometric function ${_2F_1(a,b;c;x)}$, but for the limits needed here the results simplify to
\begin{equation}
f_{\mathrm{b},\,M}(v) = \frac{N_\ast}{\left(2\pi\sigma^2\right)^{3/2}}k_M \left(\frac{v^2}{2\sigma^2}\right)^{p_M - 3}\begin{cases}
3 \Beta\left(\dfrac{v^2}{2\sigma^2}; 3 - p_M, 1 + p_M\right) & \dfrac{v^2}{2\sigma^2} \leq 1 \\
3 \Beta\left(3 - p_M, 1 + p_M\right) & \dfrac{v^2}{2\sigma^2} \geq 1
\end{cases},
\end{equation}
where $\Beta(x;a,b)$ is the incomplete beta function~\citep[8.17]{Olver2010}, $\Beta(a,b) \equiv \Beta(1,a,b)$ is the complete beta function, and $\Gamma(x)$ is the gamma function.

The velocity space density is related to the distribution by
\begin{equation}
\frac{4\pi r\sub{c}^3}{3}n_M(v_1) = 4\pi v_1^2\left[f_{\mathrm{u},\,M}(v_1) + f_{\mathrm{b},\,M}(v_1)\right].
\end{equation}

\subsection{Defining the relaxation time-scale}

Using the specific forms for the velocity space density, we can calculate the squared change in energy. The functional form depends upon the velocity of the test star. If $v_2^2/2\sigma^2 < 1$, then
\begin{align}
\Delta E^2 \approx {} & \frac{8}{3}\sqrt{2\pi}\frac{G^2m_1^2 m_2^2n_\ast}{\sigma^3}\ln\left(qv_2^2\right)\left\{\intd{0}{v_2}{3k\frac{v_1^4}{v_2}\left(\frac{v_1^2}{2\sigma^2}\right)^{p-3} \Beta\left(\frac{v_1^2}{2\sigma^2};3 - p, 1 + p\right)}{v_1} \right. \nonumber\\*
 & + \left. \intd{v_2}{\sqrt{2}\sigma}{3kv_1v_2^2\left(\frac{v_1^2}{2\sigma^2}\right)^{p-3} \Beta\left(\frac{v_1^2}{2\sigma^2};3 - p, 1 + p\right)}{v_1} \right. \nonumber\\*
 & + \left. \intd{\sqrt{2}\sigma}{\infty}{v_1v_2^2\left[3 k\left(\frac{v_1^2}{2\sigma^2}\right)^{p-3}\Beta\left(3 - p_M, 1 + p_M\right) + C\epsilon\left(\frac{v_1^2}{2\sigma^2}\right)\right]}{v_1}\right\}\,\delta t,
\end{align}
where we have suppressed the subscript $M$ for brevity; it would be necessary to sum over all the species to get the total value. Computing the integrals and rearranging yields
\begin{align}
\Delta E^2 \approx {} & \frac{16}{3}\sqrt{2\pi}\frac{G^2m_1^2 m_2^2n_\ast}{\sigma^3}\ln\left(qv_2^2\right) \left(\frac{v_2^2}{2\sigma^2}\right) \left[k \frac{3}{(2 - p)(1 + p)}{_3F_2}\left(-1-p,2-p,\frac{3}{2};3-p,\frac{5}{2};w\right) + C\right]\,\delta t,
\end{align}
where ${_3F_2}(a_1,a_2,a_3;b_1,b_2;x)$ is a generalised hypergeometric function~\citep[section 16]{Olver2010}. The contribution from bound and unbound stars can be identified by the coefficients $k$ and $C$ respectively.

If $v_2^2/2\sigma^2 > 1$,
\begin{align}
\Delta E^2 \approx {} & \frac{8}{3}\sqrt{2\pi}\frac{G^2m_1^2 m_2^2n_\ast}{\sigma^3}\ln\left(qv_2^2\right)\left\{\intd{0}{\sqrt{2}\sigma}{3k\frac{v_1^4}{v_2}\left(\frac{v_1^2}{2\sigma^2}\right)^{p-3} \Beta\left(\frac{v_1^2}{2\sigma^2};3 - p, 1 + p\right)}{v_1} \right. \nonumber\\*
 & + \left. \intd{\sqrt{2}\sigma}{v_2}{\frac{v_1^4}{v_2}\left[3k\left(\frac{v_1^2}{2\sigma^2}\right)^{p-3}\Beta\left(3 - p, 1 + p\right) + C \epsilon\left(\frac{v_1^2}{2\sigma^2}\right)\right]}{v_1} \right. \nonumber\\*
 & \left. + \intd{v_2}{\infty}{v_1v_2^2\left[3k\left(\frac{v_1^2}{2\sigma^2}\right)^{p-3}\Beta\left(3 - p_M, 1 + p_M\right) + C \epsilon\left(\frac{v_1^2}{2\sigma^2}\right)\right]}{v_1}\right\}\,\delta t.
\end{align}
Computing this gives
\begin{equation}
\Delta E^2 \approx \frac{16}{3}\sqrt{2\pi}\frac{G^2m_1^2 m_2^2n_\ast}{\sigma^3}\ln\left(qv_2^2\right) \left(\frac{v_2^2}{2\sigma^2}\right)^{-1/2} \left[k\beta\left(\frac{v_2^2}{2\sigma^2};p\right) + C\alpha\left(\frac{v_2^2}{2\sigma^2}\right)\right]\,\delta t,
\end{equation}
where
\begin{align}
\beta(w;p) = {} & \begin{cases} \dfrac{9\sqrt{\pi}\Gamma(1+p)}{2(1-2p)\Gamma(p + 7/2)} - \dfrac{9\Beta\left(3 - p, 1 + p\right)}{(1-2p)(2-p)}w^{p-1/2} & p < \recip{2} \\
\dfrac{\pi}{32}\left[12 \ln(2) - 1 + 6 \ln(w)\right] & p = \recip{2} \end{cases};\\ 
\alpha(w) = {} & \recip{2}\left\{3w^{-1/2} + 5 + \left[4\exp(1) - \Ei(1) + \Ei(w)\right]\left[\frac{3\sqrt{\pi}}{4}\erf\left(w^{1/2}\right) - \frac{3}{2}w^{1/2}\exp(-w)\right] - 3\sqrt{\pi}\exp(1)\erf(1) \right. \nonumber\\*
 & + \left. 3\left[{_2F_2}\left(\frac{1}{2},1;\frac{3}{2},\frac{3}{2};1\right) - w^{1/2}{_2F_2}\left(\frac{1}{2},1;\frac{3}{2},\frac{3}{2};w\right)\right]\right\}.
\end{align}
The hypergeometric function originates from the integral
\begin{equation}
\intd{}{w}{\frac{\exp(w')\erf\left({w'}^{1/2}\right)}{w'}}{w'} = \frac{4w^{1/2}}{\sqrt{\pi}}{_2F_2}\left(\frac{1}{2},1;\frac{3}{2},\frac{3}{2};w\right).
\end{equation}

Combining the two regimes gives some quite complicated expressions. It is possible to simplify these while maintaining reasonable accuracy, for this purpose we will separate the contributions from bound and unbound stars. The bound contribution is relatively straightforward for larger $v_2$; for smaller $v_2$ we may use a series expansion in $v_2^2/2\sigma^2 < 1$ to approximate the hypergeometric function:
\begin{align}
\Delta E\sub{b}^2 \approx {} & \frac{16}{3}\sqrt{2\pi}G^2m_1^2m_2^2n_\ast\sigma\ln\left(qv_2^2\right) k \left\{\begin{array}{ll}
\dfrac{3}{(1 + p)(2 - p)}\left(\dfrac{v_2^2}{2\sigma^2}\right) - \dfrac{9}{5(3-p)}\left(\dfrac{v_2^2}{2\sigma^2}\right)^2 + \dfrac{9p}{14(7-p)}\left(\dfrac{v_2^2}{2\sigma^2}\right)^3 & \dfrac{v_2^2}{2\sigma^2} < 1 \\
\left(\dfrac{v_2^2}{2\sigma^2}\right)^{-1/2}\beta\left(\dfrac{v_2^2}{2\sigma^2};p\right) & \dfrac{v_2^2}{2\sigma^2} > 1\end{array}\right\}\,\delta t.
\end{align}
This can be rolled into one continuous function
\begin{align}
\Delta E\sub{b}^2 \approx {} & \frac{16}{3}\sqrt{2\pi}G^2m_1^2m_2^2n_\ast\sigma\ln\left(qv_2^2\right) k \left[1 + \left(\frac{v_2^2}{2\sigma^2}\right)^4\right]^{-1}\left\{\left[\dfrac{3}{(1 + p)(2 - p)}\left(\dfrac{v_2^2}{2\sigma^2}\right) - \dfrac{9}{5(3-p)}\left(\dfrac{v_2^2}{2\sigma^2}\right)^2 + \dfrac{9p}{14(7-p)}\left(\dfrac{v_2^2}{2\sigma^2}\right)^3 \right]\right. \nonumber\\*
 & + \left. \left(\frac{v_2^2}{2\sigma^2}\right)^{7/2}\beta\left(\frac{v_2^2}{2\sigma^2};p\right)\right\}\,\delta t\\
 \approx {} & 16\sqrt{2\pi}G^2m_1^2m_2^2n_\ast\sigma\ln\left(qv_2^2\right) k \gamma\left(\frac{v_2^2}{2\sigma^2};p\right)\,\delta t,
\end{align}
defining the function $\gamma(w;p)$ in the last line. The resulting error [ignoring variation from $\ln\left(qv_2^2\right)$] is less than $3\%$.

The unbound contribution is very simple for small $v_2$, but much more complicated for large $v_2$. In the limit of $v_2 \rightarrow \infty$, it decays as $v_2^{-1}$:
\begin{align}
\lim_{w \rightarrow \infty}\left\{\alpha(w)\right\} = {} & \recip{2}\left\{5 + 3\sqrt{\pi}\left[\exp(1) - \frac{\Ei(1)}{4} - \exp(1)\erf(1)\right] + 3{_2F_2}\left(\frac{1}{2},1;\frac{3}{2},\frac{3}{2};1\right)\right\} \\
 = {} & \Xi.
\end{align}
Using the two limiting forms, the function
\begin{equation}
\Delta E\sub{u}^2 \approx \frac{16}{3}\sqrt{2\pi}G^2m_1^2m_2^2n_\ast\sigma\ln\left(qv_2^2\right) C \Xi\left(\frac{v_2^2}{2\sigma^2}\right)\left[\Xi^2 + \left(\frac{v_2^2}{2\sigma^2}\right)^3\right]^{-1/2}\,\delta t,
\end{equation}
reproduces the full function to better than $5\%$ ignoring variation from $\ln\left(qv_2^2\right)$].

Making explicit the sum over the different species, the total change in energy squared is approximately
\begin{equation}
\Delta E^2 \approx \sum_M \frac{16}{3}\sqrt{2\pi}G^2m_M^2m_2^2n_\ast\sigma\ln\left(qv_2^2\right) \left\{k_M \gamma\left(\frac{v_2^2}{2\sigma^2};p_M\right) + C_M\Xi\left(\frac{v_2^2}{2\sigma^2}\right)\left[\Xi^2 + \left(\frac{v_2^2}{2\sigma^2}\right)^3\right]^{-1/2}\right\}\,\delta t.
\end{equation}
The relaxation time-scale is the time interval $\delta t$ over which the squared change in energy becomes equal to the kinetic energy of the test star squared
\begin{align}
\tau\sub{R} = {} & \left(\frac{m_2v_2^2}{2}\right)^2\frac{\delta t}{\Delta E^2} \\
 \approx {} & \frac{3v_2^4}{16\sqrt{2\pi}G^2n_\ast\sigma\ln\left(qv_2^2\right)} \left(\sum_M m_M^2 \left\{k_M \gamma\left(\frac{v_2^2}{2\sigma^2};p_M\right) + C_M\Xi\left(\frac{v_2^2}{2\sigma^2}\right)\left[\Xi^2 + \left(\frac{v_2^2}{2\sigma^2}\right)^3\right]^{-1/2}\right\}\right)^{-1}.
\end{align}
This is the time required for stellar encounters to become effective in changing the energy of an orbit in the smooth background potential. The use of the squared change in energy reflects the expectation that energy changes like a random walk, and hence scales with the square-root of the time.

\subsection{Averaged time-scale}

The relaxation time-scale defined above is appropriate for a particular initial test star velocity $v_2$. This is not of much use to describe the Galactic centre or even a (non-circular) orbit where there is a range in velocity. It is necessary to calculate an averaged time-scale. Both the change in energy squared and the kinetic energy are averaged; comparing these gives the appropriate timescale. We will use two averages: over the distribution of bound velocities to give the relaxation time-scale for the system, and over a single orbit.

\subsubsection{System relaxation time-scale}

The total number of bound stars in the core is
\begin{equation}
N_{\mathrm{b},\,M} = \frac{3}{3/2 - p_M}\frac{\Gamma(p_M + 1)}{\Gamma(p_M + 7/2)}N_\ast k_M.
\end{equation}
Using this as a normalisation constant, the probability of a bound star having a velocity in the range $v \rightarrow v + \dd v$ is given by
\begin{align}
4\pi v^2 p_{\mathrm{b},\,M}(v) \,\dd v = {} & 4\pi v^2 \frac{f_{\mathrm{b},\,M}(v)}{N_{\mathrm{b},\,M}} \,\dd v \\
 = {} & \sqrt{\frac{2}{\pi}} \frac{v^2}{\sigma^3} \frac{\left(3/2 - p_M\right)\Gamma(p_M + 7/2)}{\Gamma(p_M + 1)} k_M \left(\frac{v^2}{2\sigma^2}\right)^{p_M - 3}\left\{\begin{array}{ll}
\Beta\left(\dfrac{v^2}{2\sigma^2}; 3 - p_M, 1 + p_M\right) & \dfrac{v^2}{2\sigma^2} \leq 1 \\
\Beta\left(3 - p_M, 1 + p_M\right) & \dfrac{v^2}{2\sigma^2} \geq 1\end{array}\right\}\,\dd v.
\end{align}
The mean squared velocity for bound stars in the core is then
\begin{align}
\overline{v^2_{M}} = {} & 4\pi\intd{0}{\infty}{v^4 p_{\mathrm{b},\,M}(v)}{v} \\
 = {} & \frac{4}{\sqrt{\pi}}\sigma^2\frac{3/2 - p_M}{1/2 - p_M}\frac{\Gamma(p_M + 7/2)}{\Gamma(1 + p_M)}\Beta\left(\frac{5}{2},1 + p_M\right),
\end{align}
assuming that $p_M < 1/2$.

In the case $p_M = 1/2$ the integral no longer converges, we encounter a logarithmic divergence. This is not a concern, just as with the divergence encountered in \secref{time-scale}, it reflects there being a physical cut-off; in this case there is a maximal velocity. Will will use $v\sub{max} = c/2$, which is the maximum speed reached on a bound orbit about a Schwarzschild BH. Marginally higher speeds can be reached for prograde orbits about a Kerr BH, but the maximal velocity for retrograde orbits is marginally lower. In reality, we might expect the maximum velocity to be lower due to a depletion of orbits (for example because of gravitational wave inspiral). We will neglect these possible variations, since the error introduced should be small as a consequence of taking the logarithm. We also suspect that a simple Newtonian description of these orbits is imprecise; however, a full relativistic description is beyond this simple analysis. For $p_M = 1/2$
\begin{align}
\overline{v^2_{M}} = {} & \frac{\sigma^2}{2}\left[12\ln(2) - 5 + 6 \ln\left(\frac{v\sub{max}^2}{2\sigma^2}\right)\right].
\end{align}

The average of the change in energy is more involved. First we will replace the term $\ln\left(qv_2^2\right)$ in $\Delta E^2$ by a suitable average so that it may be moved outside the integral~\citep[chapter 2]{Chandrasekhar1960}. We will assume that it may be replaced by the Coulomb logarithm, which can be approximated as~\citep{Bahcall1976}
\begin{equation}
\ln\left(q\overline{v_2^2}\right) = \ln \Lambda \simeq \ln\left(\frac{M_\bullet}{M_\star}\right).
\end{equation}


\section{Evolution of orbital parameters from gravitational wave emission}

\subsection{Bound orbits}

For bound orbits it is possible to define a gravitational wave inspiral time from the orbit-averaged change in the orbital parameters. Using the analysis of \citet{Peters1964} for Keplerian binaries, the averaged rates of change of the periapsis and eccentricity are
\begin{align}
\left\langle\diff{r\sub{p}}{t}\right\rangle = {} & -\frac{64}{5}\frac{\zeta}{r\sub{p}^3}\frac{(1 - e)^{3/2}}{(1 + e)^{7/2}}\left(1 - \frac{7}{12}e + \frac{7}{8}e^2 + \frac{47}{192}e^3\right) \\
\left\langle\diff{e}{t}\right\rangle = {} & -\frac{304}{5}\frac{\zeta}{r\sub{p}^4}\frac{e(1 - e)^{3/2}}{(1 + e)^{5/2}}\left(1 + \frac{121}{304}e^2\right),
\end{align}
where we have introduced
\begin{equation}
\zeta = \frac{G^3M_\bullet M(M_\bullet + M)}{c^5}.
\end{equation}
For a circular orbit the inspiral time from initial periapsis $r\sub{p0}$ is
\begin{equation}
\tau\sub{c}(r\sub{p0}) = \frac{5}{256}\frac{r\sub{p0}^4}{\zeta}.
\end{equation}
For an orbit of finite eccentricity ($0 < e < 1$), we can solve for the periapsis as a function of eccentricity
\begin{equation}
r\sub{p}(e) = \chi(1 + e)^{-1}\left(1 + \frac{121}{304}e^2\right)^{870/2299}e^{12/19},
\end{equation}
where $\chi$ is a constant fixed by the initial conditions: for an orbit with initial eccentricity $e_0$
\begin{equation}
\chi = (1 + e_0)\left(1 + \frac{121}{304}e_0^2\right)^{-870/2299}e_0^{-12/19}r\sub{p0}.
\end{equation}
The inspiral is complete when the eccentricity has decayed to zero. Consequently the inspiral time is~\citep{Peters1964}
\begin{equation}
\tau\sub{insp}(r\sub{p0},e_0) = \intd{0}{e_0}{\frac{15}{304}\frac{\chi^4}{\zeta}\frac{e^{29/19}}{(1-e^2)^{3/2}}\left(1 + \frac{121}{304}e^2\right)^{1181/2299}}{e}.
\end{equation}
This is best evaluated numerically, but it may be written in closed form as
\begin{equation}
\tau\sub{insp}(r\sub{p0},e_0) = \tau\sub{c}(r\sub{p0})(1 + e_0)^4\left(1 + \frac{121}{304}e_0^2\right)^{-3480/2299} F_1\left(\frac{24}{19};\frac{3}{2},-\frac{1181}{2299};\frac{43}{19};e_0^2,-\frac{121}{304}e_0^2\right),
\label{eq:Bound_inspiral}
\end{equation}
using the Appell hypergeometric function of the first kind $F_1(\alpha;\beta,\beta';\gamma;x,y)$~\citep[16.15.1]{Olver2010}.\footnote{For small eccentricities $\tau(r\sub{p0},e_0) \simeq \tau\sub{c}(r\sub{p0})[1 + 4e_0 + (273/43)e_0^2 + \order{e_0^3}]$.}

\subsection{Unbound orbits}\label{sec:Unbound}

Unbound orbits only pass through periapsis once. We therefore expect that their evolution through the loss of energy and angular momentum carried away by gravitational waves to be small. Following the approach of \citet{Turner1977} we can calculate the change in the eccentricity and periapse of an unbound Keplerian binary. The change in fractional eccentricity over an orbit, approximating the orbital parameters remain constant throughout, is
\begin{equation}
\frac{\Delta e}{e} = -\frac{608}{15}\Sigma\left[\recip{(1+e)^{5/2}}\left(1 + \frac{121}{304}e^2\right)\cos^{-1}\left(-\recip{e}\right) + \frac{(e - 1)^{1/2}}{e^2(1+e)^2}\left(\frac{67}{456} + \frac{1069}{912}e^2 + \frac{3}{38}e^4\right)\right],
\end{equation}
introducing dimensionless parameter
\begin{equation}
\Sigma = \frac{G^{5/2}M_\bullet M(M_\bullet+ M)}{c^5r\sub{p}^{5/2}}.
\end{equation}
Similarly, the fractional change in periapsis is
\begin{equation}
\frac{\Delta r\sub{p}}{r\sub{p}} = -\frac{128}{5}\Sigma\left[\recip{(1+e)^{7/2}}\left(1 - \frac{7}{12}e + \frac{7}{8}e^2 + \frac{47}{192}e^3\right)\cos^{-1}\left(-\recip{e}\right) - \frac{(e - 1)^{1/2}}{e(1 + e)^3}\left(\frac{67}{288} - \frac{13}{8}e + \frac{133}{576}e^2 - \frac{1}{4}e^3 - \frac{1}{8}e^4\right)\right].
\end{equation}
Both of these changes obtain their greatest magnitudes for large eccentricities, then
\begin{equation}
\frac{\Delta e}{e} \simeq \frac{\Delta r\sub{p}}{r\sub{p}} \simeq -\frac{16}{5}\Sigma e^{1/2}.
\end{equation}
For extreme mass-ratio binaries, as is the case here, the mass-ratio is a small quantity
\begin{equation}
\eta = M/M_\bullet \ll 1.
\end{equation}
The smallest possible periapsis is of order of the Schwarzschild radius of the MBH, such that 
\begin{equation}
r\sub{p} = \alpha\frac{GM_\bullet}{c^2}; \quad \alpha > 1.
\end{equation}
These give
\begin{equation}
\Sigma = \frac{\eta}{\alpha^{5/2}} < \eta \ll 1.
\end{equation}
Hence the changes in orbital parameters will be significant for
\begin{equation}
e \sim \frac{25}{256}\frac{\alpha^5}{\eta^2} > \frac{25}{256}\recip{\eta^2}.
\end{equation}
Such orbits should be exceedingly rare, and so it is safe to neglect inspiral for unbound orbits.

\end{onecolumn}

\bsp

\label{lastpage}

\end{document}




